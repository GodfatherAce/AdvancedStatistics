Statistical Genetics - Zygotes
 
 
Zygosity is the similarity of alleles of a gene for a trait (inherited characteristic) in an organism. If both alleles are the same, the organism is homozygous for the trait. If both alleles are different, the organism is heterozygous for that trait. If one allele is missing, it is hemizygous, and if both alleles are missing, it is nullizygous.
 
The DNA sequence of any gene can vary among individuals in the population. The various forms of a gene are called alleles, and diploid organisms generally have two alleles for each gene, one on each of the two homologous chromosomes on which the gene is present. In diploid organisms, the alleles are inherited from the individual's parents, one from the male parent and one from the female. Zygosity in general is a description of whether those two alleles have identical or different DNA sequences.
 
Homozygosity: The state of possessing two identical forms of a particular gene, one inherited from each parent.
 
For example, a girl who is homozygous for cystic fibrosis (CF) received the CF gene from both of her parents and therefore she has CF.


