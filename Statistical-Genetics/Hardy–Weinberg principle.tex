Statistical Genetics - 
 
The Hardy–Weinberg principle states that both allele and genotype frequencies in a population remain constant—that is, they are in equilibrium—from generation to generation unless specific disturbing influences are introduced. Those disturbing influences include non-random mating, mutations, selection, limited population size, "overlapping generations", random genetic drift and gene flow. It is important to understand that outside the lab, one or more of these "disturbing influences" are always in effect. That is, Hardy–Weinberg equilibrium is impossible in nature. Genetic equilibrium is an ideal state that provides a baseline to measure genetic change against.
 
Static allele frequencies in a population across generations assume: random mating, no mutation (the alleles don't change), no migration or emigration (no exchange of alleles between populations), infinitely large population size, and no selective pressure for or against any traits.
 
In the simplest case of a single locus with two alleles: the dominant allele is denoted A and the recessive a and their frequencies are denoted by p and q; freq(A) = p; freq(a) = q; p + q = 1. If the population is in equilibrium, then we will have freq(AA) = p2 for the AA homozygotes in the population, freq(aa) = q2 for the aa homozygotes, and freq(Aa) = 2pq for the heterozygotes.
 
Based on these equations, we can determine useful but difficult-to-measure facts about a population. For example, a patient's child is a carrier of a recessive mutation that causes cystic fibrosis in homozygous recessive children. The parent wants to know the probability of her grandchildren inheriting the disease. In order to answer this question, the genetic counselor must know the chance that the child will reproduce with a carrier of the recessive mutation. This fact may not be known, but disease frequency is known. We know that the disease is caused by the homozygous recessive genotype; we can use the Hardy–Weinberg principle to work backward from disease occurrence to the frequency of heterozygous recessive individuals.
 
This concept is also known by a variety of names: HWP, Hardy–Weinberg equilibrium, Hardy–Weinberg Theorem, HWE, or Hardy–Weinberg law. It was named after G. H. Hardy and Wilhelm Weinberg.

