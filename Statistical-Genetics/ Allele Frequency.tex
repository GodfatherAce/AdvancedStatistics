 
Statistical Genetics - Allele Frequency
 
Allele frequency is the proportion of all copies of a gene that is made up of a particular gene variant (allele). In other words, it is the number of copies of a particular allele divided by the number of copies of all alleles at the genetic place (locus) in a population. It can be expressed for example as a percentage. In population genetics, allele frequencies are used to depict the amount of genetic diversity at the individual, population, and species level. It is also the relative proportion of all alleles of a gene that are of a designed type.
Given the following:
a particular locus on a chromosome and the gene occupying that locus
a population of N individuals carrying n loci in each of their somatic cells (e.g. two loci in the cells of diploid species, which contain two sets of chromosomes)
different alleles of the gene exist
one allele exists in a copies
then the allele frequency is the fraction or percentage of all the occurrences of that locus that is occupied by a given allele and the frequency of one of the alleles is a/(n*N).

