\subsection{Bayesian Average}
A Bayesian average is a method of estimating the mean of a population consistent with Bayesian interpretation, where instead of estimating the mean strictly from the available data set, other existing information related to that data set may also be incorporated into the calculation in order to minimize the impact of large deviations, or to assert a default value when the data set is small.

Calculating the Bayesian average uses the prior mean ''m'' and a constant;''C''. ''C'';is assigned a value that is proportional to the typical data set size. The value is larger when the expected variation between data sets (within the larger population) is small. It is smaller, when the data sets are expected to vary substantially from one another.

\[  \bar{x} = {Cm + \sum_{i=1}^n{x_i} \over C + n}  \]

In cases where the averages' relative values are the only result of importance, ''m'' can be replaced with zero. ''C'';can be calculated based on the priors regarding variance between data sets. In circumstances where that kind of rigor is desired, other more expressive measures of [[statistical power]] are likely to be used. As a result, ''C'' is usually assigned a value in an ad hoc manner.

The goal is to calculate the Bayesian average of the heights of various occupations of adult American men. In the larger population of adult American men, the [[average height]] ''m'' is 176;cm. A value of ''C'' is chosen as 10. For the purpose of this example, the occupations used will be "Basketball Players", "Actors" and "Students". For the basketball players, a group of 15 individuals are identified with an average height of 191;cm among them. For the students, a group of 10 individuals is identified with an average height of 179;cm. For the actors, only James Cromwell is available, for an average height of 201;cm.


\begin{tabular}{|c|c|c|c|}
	\hline Group & N & Group Mean & Bayesian Average \\ 
	\hline Bastketball Players & 15 & 191 cm & 185 cm \\ 
	\hline Students & 10 & 179 cm & 177.5cm \\ 
	\hline Actors & 1 & 201 cm & 178 cm \\ 
	\hline 
\end{tabular} 
Here, the Bayesian average correctly reduces the effect of a single anomalously large value. If the sample sizes for basketball players were similarly small, the Bayesian average would have mis-estimated basketball players as being far closer to average.

%- http://masanjin.net/blog/bayesian-average
Is this arbitarary? Actually, no. It’s the mean (i.e. MLE) of the posterior distribution you get when you have a Normal prior with mean C  and precision m , and a Normal conditional with variance 1.0.


non-bayesian interpretation: it’s L2 regularization for the mean estimation :)
