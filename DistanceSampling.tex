Distance sampling
The term 'distance sampling' covers a range of methods for assessing wildlife abundance:
\begin{itemize}
\item	line transect sampling, in which the distances sampled are distances of detected objects (usually animals) from the line along which the observer travels 
\item	point transect sampling, in which the distances sampled are distances of detected objects (usually birds) from the point at which the observer stands 
\item	cue counting, in which the distances sampled are distances from a moving observer to each detected cue given by the objects of interest (usually whales) 
\item	trapping webs, in which the distances sampled are from the web centre to trapped objects (usually invertebrates or small terrestrial vertebrates) 
\item	migration counts, in which the 'distances' sampled are actually times of detection during the migration of objects (usually whales) past a watch point 
\end{itemize}
