Biplot
The biplot is one of the most useful and versatile methods for visualizing and interpreting multivariate data. Whenever research observations are recorded in the form of a rectangular table, such as a table stored in a spreadsheet or in a database, a biplot is usually possible to display the data in a graphical representation to facilitate understanding of the data and to reduce its complexity. The biplot extends the idea of a simple scatterplot of two variables to the case of many variables, with the objective of visualizing the maximum possible amount of information in the data.

This workshop explains how the biplot is defined in many different areas of multivariate analysis, notably regression, generalized linear modelling, principal component analysis, log-ratio analysis, various forms of correspondence analysis and discriminant analysis. In each situation the interpretation of the biplot is similar, but adapted to the different data types and objectives of the analysis.

The orientation of the workshop is towards understanding how the biplot functions and how it is applied in practice. Because of the universality and varied application of this methodology, the workshop is aimed at a wide audience of researchers in all disciplines, as well as statisticians and computer scientists interested in this subject. During the workshop applications will be presented in many different fields of the social and natural sciences, including three detailed case studies where the biplot reveals structure in large complex data sets in genomics (where thousands of variables are commonly encountered), in social survey research (where many categorical variables are studied simultaneously) and ecological research (where relationships between two sets of variables are investigated).

Computers will be provided for use during the practical sessions and participants are urged to bring their own data sets. The free software program R will be used throughout the course, and participants are expected to have a reasonable introductory knowledge of this program (for example, writing and editing programs, importing data, installing packages, and producing graphical output).  Participants are also expected to have a working knowledge of introductory statistics, up to linear regression. 
