

%--------------------------------------------------------------------------------------%
\section{Chi Square}


$p_{i}$ = Expected proportion for digit $i$.

For this test we used the chi-squared test statistic which is given by:
\begin{equation}
X^2 = \sum_{i=1}^{n} {(O_i - E_i)^2 \over E_i}
\end{equation}


The Chi Square test tests a null hypothesis stating that the frequency distribution of certain events observed in a sample is consistent with a particular theoretical distribution. The events considered must be mutually exclusive and have total probability 1. A common case for this is where the events each cover an outcome of a categorical variable.

\begin{itemize}
\item $X^2$ = the test statistic that asymptotically approaches a $\chi^2$ distribution.
\item $O_i$ = an observed frequency;
\item $E_i$ = an expected (theoretical) frequency, asserted by the null hypothesis;
\item $n $  = the number of possible outcomes of each event.
\end{itemize}

The chi-square statistic can then be used to calculate ap-value by comparing the value of the statistic to a chi-square distribution. The number of degrees of freedom is equal to the number of cells ``n'', minus the reduction in degrees of freedom, ``p''.
\subsection{Chi Square example}
In reading a burette to 0.01ml the final figure has to be estimated.
The following frequency table gives the final figures of 40 such readings.

\begin{tabular}{|c|c|c|c|c|c|c|c|c|}
\hline
Digit & 0 & 1 & 2 & 3 & 4 & 5 & 6 & 7 \\
Frequency& 1 & 6 & 4 & 5 & 3 & 11 & 2 & 8 \\
\hline
\end{tabular}

The null hypothesis is that each digit has equal chances of occurring. Since we have ten digits this implies that each digit should have a 12.5\% chance of occurring.

Mathematically we can express the null hypothesis as:

$H_{0}: p_{0} = p_{1} = p_{1}= \dots = p_{7} = 0.125$

\begin{eqnarray}
X^2
=\frac{(1 - 5)^2}{5} + \frac{(6 - 5)^2}{5} +\frac{(4 - 5)^2}{5} + \frac{(5 - 5)^2}{5}
 + \frac{(3 - 5)^2}{5}+ \frac{(11 - 5)^2}{5}+ \frac{(2 - 5)^2}{5}+
 \frac{(8 - 5)^2}{5}
\end{eqnarray}


\subsection{ Goodness of fit example}


Pressure readings are taken regularly from a meter. It transpires that, in a random
sample of 100 such readings, 45 are less than 1, 35 are between 1 and 2, and 20 are
between 2 and 3.

Perform a $\chi^2$ goodness of fit test of the model that states that the readings are
independent observations of a random variable that is uniformly distributed on (0, 3).


\subsection{Chi Square contingency tables}


In a survey the samples from five factories were examined for the
number of skilled or unskilled workers.


\begin{tabular}{|c|c|c|}
  \hline
  % after \\: \hline or \cline{col1-col2} \cline{col3-col4} ...
  Factory & skilled workers & unskilled workers \\ \hline
  A & 80 & 184 \\
  B & 58 & 147 \\
  C & 114 & 276 \\
  D & 55 & 196 \\
  E & 83 & 229 \\
  \hline
\end{tabular}

Does the population proportion of skilled and unskilled workers vary with the factory?

%--------------------------------------------------------------------------------------%
