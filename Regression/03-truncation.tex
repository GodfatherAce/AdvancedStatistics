One example of truncated samples come from historical military height records. Many armies imposed a minimum height requirement (MHR) on soldiers. This implies that men shorter than the MHR are not included in the sample. This implies that samples drawn from such records are perforce deficient i.e., incomplete, inasmuch as a substantial portion of the underlying population's height distribution is unavailable for analysis. Consequently, without proper statistical correction, any results obtained from such deficient samples, such as means, correlations, or regression coefficients are wrong (biased). In such a case truncated regression has the considerable advantage of immediately providing consistent and unbiased estimates of the coefficients of the independent variables, as well as their standard errors, thereby allowing for further statistical inference, such as the calculation of the t-values of the estimates.
