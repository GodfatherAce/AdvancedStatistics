 
Secular Trend.
A linear association (trend) with time.
Selection Bias.
 
A systematic tendency for a sampling procedure to include and/or exclude units of a certain type. For example, in a quota sample, unconscious prejudices or predilections on the part of the interviewer can result in selection bias.
 
Selection bias is a potential problem whenever a human has latitude in selecting individual units for the sample; it tends to be eliminated by probability sampling schemes in which the interviewer is told exactly whom to contact (with no room for individual choice).
 
Self-Selection.
Self-selection occurs when individuals decide for themselves whether they are in the control group or the treatment group.
 
Self-selection is quite common in studies of human behavior. For example, studies of the effect of smoking on human health involve self-selection: individuals choose for themselves whether or not to smoke.
 
Self-selection precludes an experiment; it results in an observational study. When there is self-selection, one must be wary of possible confounding from factors that influence individuals' decisions to belong to the treatment group.
 
Studentized score
The observed value of a statistic, minus the expected value of the statistic, divided by the estimated standard error of the statistic.
