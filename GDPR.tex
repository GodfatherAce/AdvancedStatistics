Content[edit]
The proposal for the European Data Protection Regulation contains the following key changes:[6] [8]

Scope[edit]
The regulation applies if the data controller or processor (organization) or the data subject (person) is based in the EU. Furthermore (and unlike the current Directive) the Regulation also applies to organizations based outside the European Union if they process personal data of EU residents. The regulation does not apply to the processing of personal data for national security activities or law enforcement ("competent authorities for the purposes of prevention, investigation, detection or prosecution of criminal offences or the execution of criminal penalties"). According to the European Commission "personal data is any information relating to an individual, whether it relates to his or her private, professional or public life. It can be anything from a name, a photo, an email address, bank details, posts on social networking websites, medical information, or a computer’s IP address." [9]

Single set of rules and one-stop shop[edit]
A single set of rules will apply to all EU member states. Each member state will establish an independent Supervisory Authority (SA) to hear and investigate complaints, sanction administrative offences, etc. SAs in each member state will cooperate with other SAs, providing mutual assistance and organising joint operations. Where a business has multiple establishments in the EU, it will have a single SA as its “lead authority”, based on the location of its "main establishment" (i.e., the place where the main processing activities take place). The lead authority will act as a “one-stop shop” to supervise all the processing activities of that business throughout the EU[10][11] (Articles 46 - 55 of the GDPR). A European Data Protection Board (EDPB) will coordinate the SAs. EDPB will replace Article 29 Working Party.

There are exceptions for data processed in an employment context and data processed for the purposes of the national security, that still might be subject to individual country regulations (Articles 2(2)(a) and 82 of the GDPR).

Responsibility and accountability[edit]
The notice requirements remain and are expanded. They must include the retention time for personal data and contact information for data controller and data protection officer has to be provided.

Automated individual decision-making, including profiling (Article 22) is made contestable. Citizens now have the right to question and fight decisions that affect them that have been made on a purely algorithmic basis.

Privacy by Design and by Default (Article 25) require that data protection is designed into the development of business processes for products and services.

Privacy settings must be set at a high level by default.

Data Protection Impact Assessments (Article 35) have to be conducted when specific risks occur to the rights and freedoms of data subjects. Risk assessment and mitigation is required and a prior approval of the Data Protection Authorities (DPA) for high risks. Data Protection Officers (Articles 37–39) are to ensure compliance within organizations. They have to be appointed for all public authorities and for companies processing more than 5000 data subjects within 12 months.

Consent[edit]
Valid consent must be explicit for data collected and purposes data used (Article 7; defined in Article 4). Consent for children [12] must be given by child’s parent or custodian, and verifiable (Article 8). Data controllers must be able to prove "consent" (opt-in) and consent may be withdrawn.[13]

Data Protection Officer[edit]
Where the processing is carried out by a public authority, except for courts or independent judicial authorities when acting in their judicial capacity, or where, in the private sector, processing is carried out by a controller whose core activities consist of processing operations that require regular and systematic monitoring of the data subjects, a person with expert knowledge of data protection law and practices should assist the controller or processor to monitor internal compliance with this Regulation. The DPO is similar but not the same as a Compliance Officer as they are also expected to be proficient at managing IT processes, data security (including dealing with cyber-attacks) and other critical business continuity issues around the holding and processing of personal and sensitive data. The skill set required stretches beyond understanding legal compliance with data protection laws and regulations. Monitoring of DPOs will be the responsibility of the Regulator rather than the Board of Directors of the organisation that employs the DPO. The appointment of a DPO within a large organisation will be a challenge for the Board as well as for the individual concerned. There are a myriad of governance and human factor issues that organisations and companies will need to address given the scope and nature of the appointment. In addition, the post holder will need to create their own support team and will also be responsible for their own continuing professional development as they need to be independent of the organisation that employs them, effectively as a "mini-regulator".

Data breaches[edit]
Under the GDPR, the independent Data Protection Officer (DPO) will be under a legal obligation to notify the Supervisory Authority without undue delay and this is also still subject to negotiations at present. The reporting of a data breach is not subject to any de minimis standard and it is likely that the GDPR will provide that such breaches must be reported to the Supervisory Authority as soon as they become aware of the data breach (Article 31). Individuals have to be notified if adverse impact is determined (Article 32).

Sanctions[edit]
The following sanctions can be imposed:

a warning in writing in cases of first and non-intentional non-compliance
regular periodic data protection audits
a fine up to 10,000,000 EUR or up to 2% of the annual worldwide turnover of the preceding financial year in case of an enterprise, whichever is greater (Article 83, Paragraph 4 [14]))
a fine up to 20,000,000 EUR or up to 4% of the annual worldwide turnover of the preceding financial year in case of an enterprise, whichever is greater (Article 83, Paragraph 5 & 6[14])
Right to erasure[edit]
A so-called right to be forgotten was replaced by a more limited right to erasure in the version of the GDPR adopted by the European Parliament in March 2014.[15][16] Article 17 provides that the data subject has the right to request erasure of personal data related to him on any one of a number of grounds including non-compliance with article 6.1 (lawfulness) that includes a case (f) where the legitimate interests of the controller is overridden by the interests or fundamental rights and freedoms of the data subject which require protection of personal data (see also Google Spain SL, Google Inc. v Agencia Española de Protección de Datos, Mario Costeja González).

Data portability[edit]
A person shall be able to transfer their personal data from one electronic processing system to and into another, without being prevented from doing so by the data controller. In addition, the data must be provided by the controller in a structured and commonly used electronic format. The right to data portability is provided by Article 18 of the GDPR. [8] Legal experts see in the final version of this measure a "new right" created that "reaches beyond the scope of data portability between two controllers as stipulated in Article 18."[17]
