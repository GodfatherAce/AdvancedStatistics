Markov Chains
MS4024 In general a chain is a discrete time process (time progressing in unit steps) in which a random variable X_n undergoes a sequence of change at a sequence of times or steps. The times or steps can be variable in duration.
 
Transitiion probability
Absolute probability
In a a certain region the weather patterns have the following sequence. A day is described as sunny (S) if the sun shines for more than 50% of daylight hours and cloudy if the sun shines less than  50% of daylight hours.
 
Data indicates that that if it is cloudy one then it is equally likely to be cloudy or sunny on the following day. If it sunny  there is a 1/3 probability that it will be cloudy and 2/3 probability that it will be sunny the next day.
The transition matrix for this process can be written as follows.
Suppose that the first day of a 60 day period is sunny.
Lets create a vector “Wthr” , that contains 60 items, one for each day of our 60 day period. Initially these values will be zero.
Wthr =numeric(60)
We will assign the value “1” to days that are sunny, and the value”2” to days that are cloudy.
To assign a value for each item, we use the “runif()” to generate a probability value.
Suppose that the current day is cloudy. If the “runif” command returns a value less than 0.5,  The outlook for tomorrow is clody. If the probability value is greater than 0.5 tomorrows weather is sunny.
 
On the other hand, if today is sunny, and the runif command generates a value less than 0.4, tomorrows weather will be cloudy. If the runif command is greater than 0.4, the weather will be sunny.
What has been the weather like for the sixty day period?
table(wthr)

 
Random Walk


First Return of a symmetric random walk.
A walk can be represented as  a connected graph between co-ordinates (n,y) where ordinate y is the position of the walk and the abscissa n represents the number of steps.
 
