\documentclass[serif]{beamer} % Get Computer Modern math font.
\hypersetup{colorlinks,linkcolor=,urlcolor=red}
\usepackage{amsmath} % Supposedly unnecessary wiht Beamer
%  To create handout using article mode: Comment above and uncomment below (2 places)
%\documentclass[12pt]{article}
%\usepackage{beamerarticle}
%\usepackage[colorlinks=true, pdfstartview=FitV, linkcolor=blue, citecolor=blue, urlcolor=red]{hyperref} % For live Web links with href in article mode
% \usepackage{amsmath} % For \binom{n}{y}
%\usepackage{graphicx} % To include pdf files!
%\usepackage{fullpage}

\usefonttheme{serif} % Looks like Computer Modern for non-math text -- nice!
\setbeamertemplate{navigation symbols}{} % Suppress navigation symbols
% \usetheme{Berlin} % Displays sections on top
\usetheme{Frankfurt}  % Displays section titles on top: Fairly thin but still swallows some material at bottom of crowded slides
%\usetheme{Berkeley}




\usepackage[english]{babel}
\usepackage{amsmath} % for binom
% \usepackage{graphicx} % To include pdf files!
% \definecolor{links}{HTML}{2A1B81}
% \definecolor{links}{red}

\setbeamertemplate{footline}[frame number] 

\mode<presentation>
% \mode<handout>{\setbeamercolor{background canvas}{bg=black!5}}


% Comment this out for handout
\title{Generalized Linear Models\footnote{See last slide for copyright information.}}
\subtitle{STA 2101/442: Fall 2012}
\date{} % To suppress date


\begin{document}

% More material required for handout in article mode. Also eliminate vspace
%\title{The Multinomial Model\footnote{See last slide for copyright information.}}
%\subtitle{STA 312: Fall 2012}
%\date{} % To suppress date
%\maketitle


\begin{frame}
  \titlepage
\end{frame}

\begin{frame}
\frametitle{Suggested Reading: Davison's \emph{Statistical models}}
%\framesubtitle{} 
  \begin{itemize}
    \item Exponential families of distributions Sec. 5.2
    \item Chapter 10 is on nonlinear regression models
    \item See pp. 468-492
  \end{itemize}

\end{frame}


% \section{Overview}

 \begin{frame}
 \frametitle{Overview}
 \tableofcontents
 \end{frame}

\section{Basics} %%%%%%%%%%%%%%%%%%%%%%%%

\begin{frame}
\frametitle{Examples of Generalized Linear Models}
%\framesubtitle{} 
  \begin{itemize}
    \item Normal regression
    \item Logistic regression
    \item Poisson regression

  \end{itemize}
\end{frame}

\begin{frame}
\frametitle{Components of a Generalized Linear Model}
%\framesubtitle{} 
  \begin{itemize}
    \item \textbf{Random Component}:  Probability distribution for $Y$
    \item \textbf{Systematic component}:  Specifies explanatory variables in the form of a ``linear predictor” that looks like a regression equation.
    \item \textbf{Link function}:  Connects $\mu = E(Y|\mathbf{X})$ to the linear predictor

  \end{itemize}
\end{frame}

\begin{frame}
\frametitle{Random Component: Distribution of $Y$}
%\framesubtitle{} 
  \begin{itemize}
    \item Ordinary regression: Normal
    \item Logistic regression: Bernoulli
    \item Poisson regression: Poisson

\vspace{20mm}

    \item Other possibilities: Binomial, Exponential, Gamma, Geometric  \ldots

  \end{itemize}
\end{frame}

\begin{frame}
\frametitle{Systematic component: A regression-like equation called the \emph{linear predictor}}
{\LARGE
\begin{displaymath}
    \eta = \beta_0 + \beta_1 x_1 + \ldots, + \beta_{p-1}x_{p-1} 
\end{displaymath}
} % End size
\end{frame}

\begin{frame}
\frametitle{Link Function: The linear predictor is an increasing function of the expected value}
%\framesubtitle{} 
{\LARGE
\begin{displaymath}
    g(\mu) = \beta_0 + \beta_1 x_1 + \ldots, + \beta_{p-1}x_{p-1} 
\end{displaymath} }

  \begin{itemize}
    \item The function $g(x)$ is strictly increasing.
    \item The linear predictor is an increasing function of $\mu$.
    \item So $\mu$ is an increasing function of the linear predictor.
  \end{itemize}


\end{frame}

\begin{frame}
\frametitle{Normal Distribution}
\framesubtitle{Link function $g(\mu) = \beta_0 + \beta_1 x_1 + \ldots, + \beta_{p-1}x_{p-1}$} 
  \begin{itemize}
    \item $E(Y)=\mu$
    \item $g(\mu) = \mu$
    \item $\mu = \beta_0 + \beta_1 x_1 + \ldots, + \beta_{p-1}x_{p-1}$
    \item The identity link
  \end{itemize}
\end{frame}

\begin{frame}
\frametitle{Bernoulli Distribution}
\framesubtitle{Link function $g(\mu) = \beta_0 + \beta_1 x_1 + \ldots, + \beta_{p-1}x_{p-1}$} 
  \begin{itemize}
    \item $E(Y)=\mu=\pi$
    \item $g(\mu) = \log \frac{\mu}{1-\mu}$
    \item $\log \frac{\mu}{1-\mu} =  \beta_0 + \beta_1 x_1 + \ldots, + \beta_{p-1}x_{p-1}$
    \item The logit link
  \end{itemize}
\end{frame}

\begin{frame}
\frametitle{Poisson Distribution}
\framesubtitle{Link function $g(\mu) = \beta_0 + \beta_1 x_1 + \ldots, + \beta_{p-1}x_{p-1}$} 
  \begin{itemize}
    \item $E(Y)=\mu=\lambda$
    \item $g(\mu) = \log(\mu)$
    \item $\log(\mu) = \beta_0 + \beta_1 x_1 + \ldots, + \beta_{p-1}x_{p-1}$
    \item The log link
  \end{itemize}
\end{frame}


\section{The Exponential Family of Distributions} %%%%%%%%%%%%%%%%%%%%%%%%

\begin{frame}
\frametitle{``Natural" Exponential Family of Distributions}
%\framesubtitle{} 
  \begin{itemize}
    \item Includes most of the familiar distributions
    \item Provides a unified theory for generalized linear models
    \item Leads to a general, highly efficient method for finding MLEs numerically
        \begin{itemize}
            \item Iterative weighted least squares
            \item Closely related to Newton-Raphson
        \end{itemize}
    \item Points to a \emph{natural} link function.
    \item The ``natural" parameter of a one-parameter exponential family is $\theta=g(\mu)$.
    \item The link functions we have been using are natural links.
  \end{itemize}
\end{frame}

\begin{frame}
\frametitle{Natural exponential family of distributions}
%\framesubtitle{} 
{\LARGE
\begin{displaymath}
    f(y|\theta,\phi) = 
    \exp\left\{ \frac{y\theta-b(\theta)}{\phi} + c(y,\phi)\right\}
\end{displaymath} }
  \begin{itemize}
    \item Support does not depend on $\theta$ or $\phi$.
    \item $\theta$ is the natural parameter.
    \item $\phi$ is the dispersion parameter, often known.
    \item $\theta=g(\mu)$, where $\mu=E(Y)$
%    \item This is the \emph{natural} link function: set $g(\mu) =  \beta_0 + \beta_1 x_1 + \ldots, + \beta_{p-1}x_{p-1}$
    \item $E(Y) = b^\prime(\theta)$ gives $\mu=g^{-1}(\theta)$
    \item $Var(Y) =  \phi \, b^{\prime\prime}(\theta) = \phi V(\mu)$
    \item $V(\mu)$ is called the \emph{variance function}.
  \end{itemize}
\end{frame}

\begin{frame}
\frametitle{Normal}
\framesubtitle{$ f(y|\theta,\phi) = 
    \exp\left\{ \frac{y\theta-b(\theta)}{\phi} + c(y,\phi)\right\}$} 

\begin{eqnarray*}
    \frac{1}{\sigma\sqrt{2\pi}} e^{\frac{(y-\mu)^2}{2\sigma^2}} 
    & = & \frac{1}{\sigma\sqrt{2\pi}}  \exp\left\{ \frac{y^2-2y\mu+\mu^2}{2\sigma^2} \right\}\\
    & = &  \exp\left\{ \frac{y\mu-\frac{\mu^2}{2}}{\sigma^2} + 
           \left( -\frac{y^2}{2\sigma^2} - \log\sqrt{\sigma^2} - \log 2\pi\right)\right\}
\end{eqnarray*} 
  \begin{itemize}
    \item Natural parameter is $\theta=\mu$
    \item Natural link is the identity function.
    \item Dispersion parameter is $\phi=\sigma^2$
    \item $b(\theta) = \frac{\theta^2}{2}$
  \end{itemize}

\end{frame}

\begin{frame}
\frametitle{Bernoulli}
\framesubtitle{$ f(y|\theta,\phi) = 
    \exp\left\{ \frac{y\theta-b(\theta)}{\phi} + c(y,\phi)\right\}$} 

\begin{eqnarray*}
    \pi^y (1-\pi)^{1-y} 
    & = & \exp\left\{ y\log\pi + (1-y)\log(1-\pi) \right\}\\
    & = &  \exp\left\{ y\left(\log\pi-\log(1-\pi)\right) + \log(1-\pi)\right\} \\
    & = &  \exp\left\{ y\left(\log\frac{\pi}{1-\pi}\right) + \log(1-\pi)\right\} \\
    & = &  \exp\left\{ \frac{y\left(\log\frac{\pi}{1-\pi}\right) - (-\log(1-\pi))}{1} + 0
\right\} \\
\end{eqnarray*} 
  \begin{itemize}
    \item Natural parameter is $\theta=\log\frac{\pi}{1-\pi} = \log\frac{\mu}{1-\mu}$
    \item Natural link is the logit function.
    \item Dispersion parameter is $\phi=1$
    \item $b(\theta) = \log(1+e^\theta)$
  \end{itemize}

\end{frame}


\section{Deviance} %%%%%%%%%%%%%%%%%%%%%%%%

\begin{frame}
\frametitle{Deviance}
%\framesubtitle{} 
  \begin{itemize}
    \item Goal is to compare a model to a ``Super" model that fits the data as well as possible.
    \item Example: If an experiment has $c$ outcomes, you can't beat a multinomial with $c$ categories. 
    \item The $c-1$ parameters soak up all $c-1$ degrees of freedom, so in this case you could call the \textbf{S}uper model ``\textbf{S}aturated."
  \end{itemize}
\end{frame}

\begin{frame} % Fix notation for 2101
\frametitle{Deviance = $-2(\ell_M-\ell_S)$}
\framesubtitle{$\ell$ is the maximized log likelihood} 
\begin{itemize}
     \item Denote the parameter of the Model by $\theta$ and the parameter of the Supermodel by $\sigma$
     \item The models might look very different, including the parameter spaces.
\end{itemize}

\begin{eqnarray*}
    -2(\ell_M-\ell_S) & = & -2log\frac{ \prod_{i=1}^n f(y_i|\widehat{\theta}) }
                                { \prod_{i=1}^n f(y_i|\widehat{\sigma}) }\\
 & = &  -2log \prod_{i=1}^n \frac{f(y_i|\widehat{\theta})}
                                 {f(y_i|\widehat{\sigma})} \\
 & = & \sum_{i=1}^n -2\log \left(\frac{f(y_i|\widehat{\theta})}
                                 {f(y_i|\widehat{\sigma})} \right) \\
 & = & \sum_{i=1}^n d_i
\end{eqnarray*} 
\end{frame}

\begin{frame}
\frametitle{Deviance = $-2log\frac{ \prod_{i=1}^n f(y_i|\widehat{\theta}) }
                                { \prod_{i=1}^n f(y_i|\widehat{\sigma}) }
                     = \sum_{i=1}^n d_i$}
%\framesubtitle{} 
  \begin{itemize}
    \item The deviance terms $d_i$ are contributions to a difference in fit (deviance) between the model and the best possible model.
    \item They are somewhat like residuals.
    \item Maybe big ones are worth investigating.
    \item Deviance residuals are defined as $r^D_i = $sign$(y_i-\widehat{\mu}_i)\sqrt{d_i}$
  \end{itemize}
\end{frame}

\begin{frame}
\frametitle{Deviance looks like the likelihood ratio statistic $G^2$}
Deviance = $-2log\frac{ \prod_{i=1}^n f(y_i|\widehat{\theta}) }
                                { \prod_{i=1}^n f(y_i|\widehat{\sigma}) }
                     = \sum_{i=1}^n d_i$
  \begin{itemize}
    \item Looks like the model represents a null hypothesis.
    \item The Supermodel is somehow less restricted.
    \item So \emph{sometimes} it must be a chi-squared test for goodness of model fit.
    \item What is that ideal ``Supermodel" that fits as well as possible?
  \end{itemize}
\end{frame}

\begin{frame}
\frametitle{What is the model that fits as well as possible?}
%\framesubtitle{} 
  \begin{itemize}
    \item If just a few ($c$) categories and plenty of observations in each category (say at least 5), the best model is a multinomial. 
        \begin{itemize}
            \item Any model with $c-1$ parameters that are 1-1 with $\pi_1, \ldots, \pi_{c-1}$ will \emph{soak up} all the degrees of freedom and is said to be ``saturated."
            \item For a saturated model, the deviance is zero.
            \item A model with fewer than $c-1$ parameters cannot be saturated, and the deviance is a likelihood ratio test statistic, null hypothesis that the model is true.
        \end{itemize}
    \item There are some other examples of super-models that are reasonable. In structural equation models, an example is the unrestricted multivariate normal.
    \item Often, the super-model is not a reasonable model.
  \end{itemize}
\end{frame}

\begin{frame}
\frametitle{An unreasonable model}
Logistic regression with continuous explanatory variables
  \begin{itemize}
    \item One observation only in each of $n$ combinations of explanatory, response variable values.
    \item One parameter for each observation.
    \item Model fits perfectly.
    \item Likelihood equals one.
    \item All parameter estimates on the boundary of the parameter space.
    \item \emph{Not} chi-squared under $H_0$.
    \item Denominator of deviance equals one.
    \item Deviance is just -2 log likelihood of the model.
    \item Deviance is not a test of model fit, or anyway nobody knows the distribution under $H_0$.
  \end{itemize}
What happens when there are \emph{a few} ties in the explanatory variable values \ldots
\end{frame}

\begin{frame}
\frametitle{R's \texttt{help glm} defines the deviance as}
``\dots up to a constant, minus twice the maximized log-likelihood. Where sensible, the constant is chosen so that a saturated model has deviance zero."

\vspace{10mm}

At least, Deviance = $-2(\ell_M-\ell_S)$ is -2log likelihood  plus a constant, so the \emph{difference} in deviance values between 2 nested models should be the large-sample likelihood ratio test of full \emph{vs.} reduced.
\end{frame}


\begin{frame}
\frametitle{One last scary question}
\framesubtitle{And a reassuring answer}
If you fit a full and a reduced model separately, might they use a different definition of the supermodel, and hence the deviance?
  \begin{itemize}
    \item I have tried unsuccessfully to make R misbehave this way.
    \item The null deviance is the deviance of a model with just an intercept.
    \item Compare the null deviance of your full and reduced models. If they are the same, both models are using the same definition of deviance and everything is okay.
    \item And in my experience with R's \texttt{glm} function, they are always the same.
  \end{itemize}
\end{frame}

%%%%%%%%%%%%%%%%%%%%%%%%%%%%%%%%%%%%%%%%%%%%%%%%%%%%%%%%%%%%%%%%%%%%%%%%%%%%%%%%%%%%%%%%%

%\section{Copyright}

\begin{frame}
\frametitle{Copyright Information}

This slide show was prepared by  \href{http://www.utstat.toronto.edu/~brunner}{Jerry Brunner},
Department of Statistics, University of Toronto. It is licensed under a 
\href{http://creativecommons.org/licenses/by-sa/3.0/deed.en_US}
     {Creative Commons Attribution - ShareAlike 3.0 Unported License}. Use any part of it as you like and share the result freely. The \LaTeX~source code is available from the course website:
\href{http://www.utstat.toronto.edu/~brunner/oldclass/appliedf12} {\texttt{http://www.utstat.toronto.edu/$^\sim$brunner/oldclass/appliedf12}}

\end{frame}

\end{document}


%%%%%%%%%%%%%%%%%%%%%%%%%%%%%%%%%%%%%%%%%%%%%%%%%%%%%%%%%%%%%%%%%%%%%%%%%%%%%%%%%%%%%%%%%

\begin{frame}
\frametitle{}
%\framesubtitle{} 
  \begin{itemize}
    \item 
    \item 
    \item 
  \end{itemize}
\end{frame}



  \begin{itemize}
    \item 
    \item 
    \item 
  \end{itemize}

%%%%%%%%%%%%%%%%%%%%%%%%%%%%%%%%%%%%%%%%%%%%%%%%%%%%%%%%%%%%%%%%%%%%%%%%%%%%%%%%%%%%%%%%%


{\tiny
\begin{center}
\begin{tabular}{|c|c|c|}  \hline
 $x$               & $a-x$          &{\color{red} $a$ } \\ \hline
 $b-x$        & $1-a-b+x$          &{\color{red} $1-a$ } \\ \hline
{\color{red} $b$ } &{\color{red} $1-b$ } & {\color{red} $1$ } \\ \hline
\end{tabular}
\end{center}
} % End small size

\begin{center}
\begin{tabular}{|l|c|c|c|c|c|}  \hline
                  &  Heart attack &  Stroke & Both & Neither & Total  \\ \hline
Drug              & $\pi_{11}$ & $\pi_{12}$ &{\color{red} $\pi_{1+}$ } \\ \hline
Drug and Exercise & $\pi_{21}$ & $\pi_{22}$ &{\color{red} $\pi_{2+}$ } \\ \hline
Total &{\color{red} $\pi_{+1}$ } &{\color{red} $\pi_{+2}$ } & {\color{red} $1$} \\ \hline
\end{tabular}
\end{center}


\begin{frame}
\frametitle{Save}
%\framesubtitle{} 

{\tiny
\begin{center}
\begin{tabular}{|l|c|c|}  \hline
      &   $Y=1$ &  $Y=2$     \\ \hline
$X=1$ & $\pi_{11}$ & $\pi_{12}$  \\ \hline
$X=2$ & $\pi_{21}$ & $\pi_{22}$  \\ \hline
\end{tabular}
\end{center}
} % End small size



{\tiny
\begin{center}
\begin{tabular}{|c|c|}  \hline
 $\pi_{11}$ & $\pi_{12}$  \\ \hline
 $\pi_{21}$ & $\pi_{22}$  \\ \hline
\end{tabular}
\end{center}
} % End small size



\end{frame}



\begin{frame}
\frametitle{R example}

Skip to the \href{http://www.utstat.toronto.edu/~brunner/312f12/lectures/312f12Independence1.pdf}
             {Independence Part One handout}.

\vspace{5mm}

If that doesn't work, 
\href{http://www.utstat.toronto.edu/~brunner/oldclass/312f12/lectures/312f12Independence1.pdf}
             {Try this}.
\end{frame}

