MS4217 Stochastic Processes: Random Walks
 
  
Random walk is a one dimensional random variable.
 
The number of different walks of n steps where each step is +1 or −1 is clearly 2n.
 
For the simple random walk, each of these walks are equally likely.
 
In order for Sn to be equal to a number k it is necessary and sufficient that the number of +1 in the walk exceeds those of -1 by k.
 
Thus, the number of walks which satisfy Sn=k is n(n+k)/2, i.e. precisely the number of ways of choosing (n + k)/2 elements from an n element set (for this to be non-zero, it is necessary that n+k be an even number) .
 
Therefore, the probability that Sn=k is equal to 2-nn(n+k)/2

%=========================================================================================================%
 

Simple random walks: 

Outcomes ={ -1,0,1}

Process jumps one step at a time.

Set up the transition matrix for a simple random walks with a partially reflecting barrier at the origin given that the states of the 
process are ={ 0,1,2,3,} (Assume pj.j+1= p)

Three types of barriers:

Absorbing barriers : once you reach this state, you cant get out of it.
Fully reflecting barriers:  you cant enter nor exit this state
Partially reflecting barriers: you can enter, but you will exit the way you entered.

Gambler's ruin is a simple random walk with absorbing barriers are states 0 and a.
Unrestricted random walks

Unrestricted random walks are processes where there are no barriers.

Three important questions:
	1) What is the probability of a return to the origin?
	2) Is a return certain?
	3) What is the distributions after n steps?






1) Return to Origin

Uk	probability of going to state 0 when starting at state k 
	[Similar to probability of ruin, from Gambler's ruin theory ]

A return to the origin is only certain if p = q = 0.5.
Uk= 1 if p = q = 0.5.

3) Probability after n steps

Xn : Position after n steps
n   : Number of steps taken
Rn : Number of rightward steps (consider a move to the right as positive)
Ln : Number of leftward steps (consider a move to the left as negative)

	necessarily
	
	 (a) n =Rn+Ln		 (b) Xn = Rn - Ln 

re-arranging (a) and (b)   Rn= 0.5(n+Xn) 

let vn,x be the probability of walk being at step x after n steps.

vn,x =  n0.5(n+x)p0.5(n+x)q0.5(n-x)


Example: 

A random walk at k, where k is a positive integer, and takes place on the integers. It is assumed that Xn+1, the location of the walk after(n+1) steps satisfies;
	Xn+1 = Xn+1 with probability p
	Xn+1 = Xn-1 with probability q = 1-p

Find the mean and variance ofXn
Comment on the result when p = q = 0.5
Derive the probability distribution Xn when k = 0.


