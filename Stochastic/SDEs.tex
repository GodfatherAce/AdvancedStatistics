Ito’s  SDE
 
 
Itō's lemma is used in Itō stochastic calculus to find the differential of a function of a particular type of stochastic process. It is the stochastic calculus counterpart of the chain rule in ordinary calculus and is best memorized using the Taylor series expansion and retaining the second order term related to the stochastic component change. The lemma is widely employed in mathematical finance and its best known application is in the derivation of the Black-Scholes equation used to value options.
 
 
The Cox-Ingersoll-Ross model for the short-term interest rate r(t).
 
The Fokker–Planck equation describes the time evolution of the probability density function of the position of a particle, and can be generalized to other observables as well. It is also known as the Kolmogorov forward equation. The first use of the Fokker–Planck equation was for the statistical description of Brownian motion of a particle in a fluid.
 
Alternating Direction Implicit (ADI)
In numerical analysis, the Alternating Direction Implicit (ADI) method is a finite difference method for solving parabolic and elliptic partial differential equations.[1] It is most notably used to solve the problem of heat conduction or solving the diffusion equation in two or more dimensions.
The traditional method for solving the heat conduction equation is the Crank–Nicolson method. This method can be quite costly. The advantage of the ADI method is that the equations that have to be solved in every iteration have a simpler structure and are thus easier to solve.
 
