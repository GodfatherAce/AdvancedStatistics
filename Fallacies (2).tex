MA4125 Paradoxes, Fallacies and Misuse of Statistics.

%=================================================================================%
 

Fallacy of Exclusion. Leaving out evidence that would lead to a different conclusion is called the fallacy of exclusion. An example is: In the presidential elections of 2000 and 2005, Florida went to Bush, so it must be a Republican state. In fact, the evidence from 1996, which I purposely excluded from the sentence above, shows that Florida went to Clinton in that election, making this, too, a fallacy of insufficient evidence. By choosing to begin with the data from 2000, I was able to exclude evidence that contradicted the conclusion I wished to draw for the sake of this exercise.
 
Fallacy of Oversimplification.
In this fallacy, some aspects of an issue -- generally more subtle ones -- and their ramifications are not explored. An example is: The question of funding medical research comes down to this: do we want to heal the sick and help the injured to recover -- or not? This argument ignores questions of funding sources, differing states of research in different areas of health care, an so on, so it falls into the category of insufficient evidence. By avoiding reference to any complexities, including the possibility that some issues may never have a successful resolution, this argument makes the choice seem to be solely about good will towards the less fortunate.


%====================================================================================%



Previously we have encountered some notable statistical fallacies, or sources of fallacies. In this class we will discuss statistical fallacies in greater detail.



1) The Monty Hall problem

Marilyn Vos Savant


2) The Post Facto Fallacy

3) The Data dredging fallacy.

4) Correlation does not imply causation

In a widely-studied example, numerous epidemiological studies showed that women who were taking combined hormone replacement therapy (HRT) also had a lower-than-average incidence of coronary heart disease (CHD), leading doctors to propose that HRT was protective against CHD. But randomized controlled trials showed that HRT caused a small but statistically significant increase in risk of CHD. Re-analysis of the data from the epidemiological studies showed that women undertaking HRT were more likely to be from higher socio-economic groups (ABC1), with better than average diet and exercise regimes. The use of HRT and decreased incidence of coronary heart disease were coincident effects of a common cause (i.e., the benefits associated with a higher socioeconomic status), rather than cause and effect as had been supposed

5) Texas sharpshooter falllacy

6) Freedman's paradox


7) The Ecological Fallacy
 

The ecological fallacy was discussed in a court challenge to the Washington
gubernatorial election, 2004 in which a number of illegal voters were identified,
after the election; their votes were unknown, because the vote was
by secret ballot. The challengers argued that illegal votes cast in
the election would have followed the voting patterns of the precincts
in which they had been cast, and thus adjustments should be made accordingly.
An expert witness said this approach was like trying to figure out Ichiro Suzuki's
batting average by looking at the batting average of the entire Seattle Mariners
team since the illegal votes were cast only by males and the overall precinct votes
included both males and females.
The judge determined that the challengers' argument was an ecological fallacy, and rejected it.


Pasting... Cancel
