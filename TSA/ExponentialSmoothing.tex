%------------------------------------------------%
\section{Exponential Smoothing}

\[
\mbox{New Forecast} = \mbox{Old Forecast} + \alpha \mbox{(Latest Observation - Old Forecast)}
\]

\begin{itemize}
\item Greater weight is given to more recent data.
\item All past data is incorporated, and there is no cut-off point as with moving averages.
\end{itemize}

%------------------------------------------------%

Exponential smoothing is commonly applied to financial market and economic data, but it can be used with any discrete 
set of repeated measurements. The simplest form of exponential smoothing should be used only for data without any 
systematic trend or seasonal components
 
The raw data sequence is often represented by $\{x_t\}$ beginning at time t=0, and the output of the exponential 
smoothing algorithm is commonly written as $\{s_t\}$, which may be regarded as a best estimate of what the next value of 
x will be. When the sequence of observations begins at time t = 0, the simplest form of exponential smoothing is given 
by the formulae:
 
%% FORMULAE HERE %%

where $\alpha$ is the smoothing factor, and $0 < \alpha < 1 $.


\end{document}
