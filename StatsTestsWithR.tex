\chapter{Inference Procedures}
%-----------------------------------------------------------------------------------------------------%
%-----------------------------------------------------------------------------------------------------%
\section{Two Sample t test}

The two-sample t test is used to test the hypothesis that two samples may
be assumed to come from distributions with the same mean.

The theory for the two-sample t test is not very different in principle from
that of the one-sample test. Data are now from two groups, $x_{11}, . . . , x_{1n1}$
and $x_{21}, . . . , x_{2n2}$ , which we assume are sampled from the normal distributions
$N(µ_{1}, \sigma^{1}_{2} )$ and
$N(µ_{2}, \sigma^{2}_{2} )$, and it is desired to test the null hypothesis
$\mu_{1} = \mu_{2}$. You then calculate

\[
t = \frac{\bar{X}_{1}-\bar{X}_{2}}{S.E.(\bar{X}_{1}-\bar{X}_{2})}
\]

%----------------------------------------------------%
\section{Chi-squared Test}

A $chi^2$ test is carried out on tabular data containing counts, e.g. the
number of animals that died, the number of days of rain, the
number of stocks that grew in value, etc.

Usually have two qualitative variables, each with a number of
levels, and want to determine if there is a relationship between the
two variables, e.g. hair colour and eye colour, social status and
crime rates, house price and house size, gender and left/right
handedness.

The data are presented in a contingency table:
right-handed left-handed TOTAL

\begin{tabular}{|c|c|c|c|}
  \hline
  % after \\: \hline or \cline{col1-col2} \cline{col3-col4} ...
  & right-handed &left-handed & TOTAL\\\hline
  Male & 43 & 9 & 52 \\
  Female & 44 & 4 & 48 \\
  TOTAL & 87 & 13 & 100 \\
  \hline
\end{tabular}


The hypothesis to be tested is
$H0 :$There is no relationship between gender and left/right-handedness
$H1 :$There is a relationship between gender and left/right-handedness
 The values that we collect from our sample are called the observed
(O) frequencies (counts). Now need to calculate the expected (E)
frequencies, i.e. the values we would expect to see in the table, if
H0 was true.






%------------------------------------------------------%
\section{Two Sample Tests}


All of the previous hypothesis tests and confidence intervals can be
extended to the two-sample case.

The same assumptions apply, i.e. data are normally distributed in
each population and we may want to test if the mean in one
population is the same as the mean in the other population, etc.

Normality can be checked using histograms, boxplots and Q-Q
plots as before. The Anderson-Darling test can be used on
each group of data also.

%------------------------------------------------------%
\section{The T test}

%------------------------------------------------------%
\section{Comparison of variances}


Even though it is possible in R to perform the two-sample t test without
the assumption that the variances are the same, you may still be interested
in testing that assumption, and R provides the var.test function for that
purpose, implementing an F test on the ratio of the group variances. It is
called the same way as \texttt{t.test}:.
\begin{verbatim}
> var.test(expend~stature)
\end{verbatim}
