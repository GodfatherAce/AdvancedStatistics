%%- http://support.sas.com/documentation/cdl/en/etsug/63939/HTML/default/viewer.htm#etsug_model_sect043.htm

Previous Page | Next Page
Testing for Normality
The NORMAL option in the FIT statement performs multivariate and univariate tests of normality.
The three multivariate tests provided are Mardia’s skewness test and kurtosis test (Mardia 1970) and the Henze-Zirkler  test (Henze and Zirkler 1990). The two univariate tests provided are the Shapiro-Wilk W test and the Kolmogorov-Smirnov test. (For details on the univariate tests, refer to "Goodness-of-Fit Tests" section in "The UNIVARIATE Procedure" chapter in the Base SAS Procedures Guide.) The null hypothesis for all these tests is that the residuals are normally distributed.
For a random sample , , where d is the dimension of  and n is the number of observations, a measure of multivariate skewness is
 		 	 
where S is the sample covariance matrix of X. For weighted regression, both S and  are computed by using the weights supplied by the WEIGHT statement or the _WEIGHT_ variable.
Mardia showed that under the null hypothesis  is asymptotically distributed as . For small samples, Mardia’s skewness test statistic is calculated with a small sample correction formula, given by  where the correction factor  is given by . Mardia’s skewness test statistic in PROC MODEL uses this small sample corrected formula.
A measure of multivariate kurtosis is given by
 		 	 
Mardia showed that under the null hypothesis,  is asymptotically normally distributed with mean  and variance .
The Henze-Zirkler test is based on a nonnegative functional  that measures the distance between two distribution functions and has the property that
 		 	 
if and only if
 		 	 
where  is a d-dimensional normal distribution.
The distance measure  can be written as
 		 	 
where  and  are the Fourier transforms of P and Q, and  is a weight or a kernel function. The density of the normal distribution  is used as 
 		 	 
where .
The parameter  depends on  as
 		 	 
The test statistic computed is called  and is approximately distributed as a lognormal. The lognormal distribution is used to compute the null hypothesis probability.
 				 	 
 				 	 
where
 		 	 
 		 	 
Monte Carlo simulations suggest that  has good power against distributions with heavy tails.
The Shapiro-Wilk W test is computed only when the number of observations (n ) is less than  while computation of the Kolmogorov-Smirnov test statistic requires at least  observations.
The following is an example of the output produced by the NORMAL option.

proc model data=test2;
   y1 = a1 * x2 * x2 - exp( d1*x1);
   y2 = a2 * x1 * x1 + b2 * exp( d2*x2);
   fit y1 y2 / normal ;
run;
Figure 19.38 Normality Test Output
The MODEL Procedure
Normality Test
Equation	Test Statistic	Value	Prob
y1	Shapiro-Wilk W	0.37	<.0001
y2	Shapiro-Wilk W	0.84	<.0001
System	Mardia Skewness	286.4	<.0001
 	Mardia Kurtosis	31.28	<.0001
 	Henze-Zirkler T	7.09	<.0001
Previous Page | Next Page |Top of Page
