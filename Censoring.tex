
\subsection{Censoring}

\begin{itemize}
\item Censoring occurs when the value of an observation is only partially known. 
\item \textbf{Survivorship} For example, suppose a study is conducted to measure the impact of a drug on mortality. 
In such a study, it may be known that an individual's age at death is at least 75 years. 
Such a situation could occur if the individual withdrew from the study at age 75, or if the individual is currently 
alive at the age of 75.
\item \textbf{Range} Censoring also occurs when a value occurs outside the range of a measuring instrument. 
\item For example, a bathroom scale might only measure up to 300 lbs. 
If a 350 lb individual is weighed using the scale, the observer would only know that the individual's weight is at 
least 300 lbs.
\end{itemize}
%-------------------------------------------------------------------------------%
\newpage
\subsection*{Types of Censorship and Truncation}
\begin{itemize}
\item Right censoring occurs when a subject leaves the study before an event occurs,
or the study ends before the event has occurred. 
\item For example, we consider
patients in a clinical trial to study the e¤ect of treatments on stroke occurrence.
The study ends after 5 years. Those patients who have had no strokes by the
end of the year are censored. If the patient leaves the study at time $t_e$; then the
event occurs in $(t_e, \infty)$ :
\item Left censoring is when the event of interest has already occurred before
enrolment. This is very rarely encountered.
\item Truncation is deliberate and due to study design.
\item Right truncation occurs when the entire study population has already
experienced the event of interest (for example: a historical survey of patients
on a cancer registry).
\item Left truncation occurs when the subjects have been at risk before entering
the study (for example: life insurance policy holders where the study starts on
a fixed date, event of interest is age at death).
\item Generally we deal with right censoring and sometimes left truncation.
\end{itemize}

%-----------------------------------------------------------------------%
\newpage
Two types of independent right censoring:
\begin{description}
\item[Type I :] completely random dropout (eg emigration) and/or fixed time of
end of study no event having occurred.
\item[Type II:] study ends when a fixed number of events amongst the subjects
has occurred.
\end{description}
\end{document}
