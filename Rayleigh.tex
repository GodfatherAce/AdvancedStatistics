


%======================================================================================================%

The Rayleigh distribution is a continuous probability distribution for positive-valued random variables.

A Rayleigh distribution is often observed when the overall magnitude of a vector is related to its directional components. One example where the Rayleigh distribution naturally arises is when wind velocity is analyzed into its orthogonal 2-dimensional vector components. 

Assuming that each component is uncorrelated, normally distributed with equal variance, and zero mean, then the overall wind speed (vector magnitude) will be characterized by a Rayleigh distribution. 

A second example of the distribution arises in the case of random complex numbers whose real and imaginary components are independently and identically distributed Gaussian with equal variance and zero mean. In that case, the absolute value of the complex number is Rayleigh-distributed.



%======================================================================================================%

\end{document}
