MODULE 6: Further applications of statistics
%============================================================================%
\subsection*{Design and analysis of experiments}
Principles of design including randomisation, blinding, pairing and blocking.
Reasons for using these.
Randomised block designs. Latin squares.
Analysis of variance, inference for means and for differences in means.
Factorial treatment structure with two factors. Advantages of factorial experimentation. Diagrammatic explanation of interaction.
Two-way analysis of variance.
Diagrams of means of treatment combinations and their use for explaining interactions of two factors.
Analysis of variance, inference for means and for differences in means.
Residuals and their use in checking assumptions.
%===========================================================================%
\subsection*{Multiple regression}
Least squares estimation for multiple regression.
Extension of material in Module 4. Derivation of normal equations in simple cases. Matrix notation will not be required. Solution of simultaneous equations in simple cases only.
Regression through the origin.
Including multiple regression with zero intercept.
Use of backwards elimination in multiple regression.
Use of F tests.
Polynomial regression.
Simple cases only.
Use of indicator variables to model factors or qualitative variables.
Simple cases only.
Residuals and their use in checking assumptions.
%===========================================================================%
\subsection*{Quality control and acceptance sampling}
Charts for mean and range for Normal data. Charts for proportions.
Construction and use of Shewhart charts, including use of warning and action lines.
Cusum charts.
Construction and use.
Attribute sampling. Single and double sampling schemes.
%==========================================================================%

In statistics, polynomial regression is a form of linear regression in which the relationship between the independent variable x and the dependent variable y is modelled as an nth order polynomial. Polynomial regression fits a nonlinear relationship between the value of x and the corresponding conditional mean of y, denoted E(y | x), and has been used to describe nonlinear phenomena such as the growth rate of tissues,[1] the distribution of carbon isotopes in lake sediments,[2] and the progression of disease epidemics.[3] Although polynomial regression fits a nonlinear model to the data, as a statistical estimation problem it is linear, in the sense that the regression function E(y | x) is linear in the unknown parameters that are estimated from the data. For this reason, polynomial regression is considered to be a special case of multiple linear regression.
\end{document}
