%MODULE 8: Survey Sampling and Estimation
Populations and frames
Target and study populations. Types of frames available, uses and sources of error.
%-----------------------------------%
\subsection*{Sampling methods}
\begin{itemize}
\item	Non-probability methods, haphazard sampling, quota sampling.
\item	Simple random sampling, stratified random sampling (with equal, proportional and optimal allocation), cluster and multi-stage sampling, systematic sampling.
\item	Advantages and drawbacks of these methods, sources of bias, unknown precision.
\item	Uses, benefits and limitations of each method. Discussion of practical and theoretical utility of methods in the context of specific examples.
\item	Simple and stratified random sampling
\item	Uses, limitations, applications to different data types, practical examples.
\item	Estimates of totals, means and proportions, construction of confidence intervals.
\item	Neyman and optimal allocation, use to reduce variance.
\item	Discussion and comparison of these methods.
\item	Calibration techniques for estimation
\item	Ratio and regression methods. Use of supplementary information.
\item	Discussion of how and why other survey information may be useful in estimation. Formulae will not be required.
\item	Practical problems in planning and conducting surveys
\item	Choice of sampling method and estimators to be used in a survey, trade-off between bias and variance. Discussion of sampling problems in an actual survey, recommendations for improvement using practical examples.
\item	Cross-sectional, longitudinal and panel surveys, drug trials, etc.
\end{itemize}
