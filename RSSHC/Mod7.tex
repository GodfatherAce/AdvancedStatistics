
\section*{MODULE 7: Time Series and Index Numbers}
%--------------------------------------------------------%
\subsection*{Time series}
\begin{itemize}
\item	Decomposition of time series into trend, cycles, seasonal variation and residual (irregular) variation.
\item	Estimation of trend using regression or moving averages.
\item	Examination of seasonal terms in time series decompositions. 
\item	Seasonal adjustment using regression or moving averages.
\item	Elementary forecasting methods: exponential smoothing and Holt-Winters.
\item	Introduction to ARIMA models.
\item	Examination and interpretation of residuals from fitted models.
\end{itemize}
%--------------------------------------------------------%
\subsection*{Interpretation of computer output.}
\begin{itemize}
\item	Additive, multiplicative and pseudo-additive decompositions.
\item	Equally and unequally weighted moving averages.
\item	Prior adjustment. Detection and control of special events in seasonal adjustment, for example religious holidays, changes in taxation levels, … .
\item	Definition, interpretation, forecasting. Details of techniques for fitting ARIMA models are not required.
\item	Including the correlogram and the portmanteau lack-of-fit test.
\item	Analysis and commentary on tables and graphs produced by seasonal adjustment programs.
\item	Confirmation of some results through hand calculation.
\end{itemize}
%--------------------------------------------------------%
\subsection*{Index numbers}
\begin{itemize}
\item	Introduction to index numbers
\item	Index numbers and their uses. Simple price relatives, Laspeyres and Paasche. Relationship between Laspeyres and Paasche; and the relative merits of each. Further index numbers – Törnqvist, Walsh, Fisher and Geometric Laspeyres. Index aggregation.
\item	Consideration of price and volume indices, with examples from candidate’s knowledge and experience.
\item	Calculation of index numbers by aggregating others.
\end{itemize}
%--------------------------------------------------------%
\subsection*{Deflation}
\begin{itemize}
\item	Why deflation is used and how it works; what makes a good deflator; how deflation is carried out.
\item	With reference to specific examples from candidate’s knowledge and experience.
\item	Re-basing
\item	Why, when and how re-basing is done.
\item	With reference to specific examples from candidate’s knowledge and experience. Benefits and pitfalls.
\item	Chain linking
\item	Chain linking of simple price relatives, and chain linking using Laspeyres.
\item	With reference to specific examples from candidate’s knowledge and experience. Benefits and pitfalls.
\item	Use of index numbers
\item	With reference to specific examples from candidate’s knowledge and experience, for example within National Accounts.
\end{itemize}

