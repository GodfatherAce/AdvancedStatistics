\section*{MODULE 3: Basic statistical methods}
\subsection*{Inference}
Sample and population. 
Concept of a sampling distribution. 
Standard error.
Sampling distribution of the mean.
%===========================================%
Point and interval estimates. 
Construction and interpretation of confidence limits.
Hypothesis tests, test statistic, one- and two-sided tests.
Significance level. 
Type I and II errors. Power as 1–P (type II error).
Knowledge of p-values and their interpretation.
%===========================================%
Use of Normal, t, $\chi^2$ and F distributions in testing and interval estimation.
Paired and unpaired two-sample tests.
Tests and confidence intervals involving means, variances and proportions. 
Use of tables to obtain percentage points.
Power curves.
Restricted to cases of testing for the mean of a Normal distribution with known variance.
The $\chi^2$ goodness-of-fit test of standard distributions to observed data.
%==========================================%
Including pooling of classes. Uniform (discrete and continuous), binomial, Poisson and Normal distributions; distributions in specified proportions.
Analysis of two-way contingency tables; $\chi^2$ test for association.
Use of Yates' correction in $2 \times 2$ tables is expected.
McNemar’s test.
Use of a continuity correction is not expected. Consideration of tied ranks will not be expected.
%==========================================%
\subsection{Non-parametric methods}
Use of non-parametric and distribution-free significance tests for paired and unpaired data: sign test, Wilcoxon rank sum test (Mann-Whitney U test), Wilcoxon signed-rank test.
Candidates will be expected to be able to use tables of percentage points but do not need to know how the tables are obtained.
Consideration of tied ranks will not be expected.
%==========================================%
\end{document}
