\section*{MODULE 1: Data collection and interpretation}
%=======================================================%
\subsection*{Summarising and interpreting data}
Frequency distributions. Numerical and graphical forms of presentation and statistical interpretation.
Scatter diagrams, time charts, stem and leaf diagrams, histograms, bar charts, pie charts, frequency and cumulative frequency curves, boxplots (box and whisker plots), dotplots.
Summary statistics for measures of location, variability and skewness.

\textbf{Remarks}\\
Credit will be given for graphs that are well presented, including neat appearance and inclusion of appropriate title and labels.
Sample mean, median, mode, quartiles, semi-inter-quartile range, standard deviation, variance, range.
%=======================================================%
\subsection*{Surveys}
Target and study populations. Sampling frames. Problems arising in the collection of data.
Censuses, sample surveys and routine collection of data at intervals of time.
Principles and practice, with examples from candidates' knowledge and experience.
Design of questionnaires and forms for collecting data.
Personal and telephone interviews, postal enquiries, pilot enquiries.
Problems of non-response, bias among interviewers, question bias, non-sampling errors.
Advantages and disadvantages of these methods.
Simple random sampling. Uses and limitations. Estimators for means, totals and proportions and the variances of these estimators.

\textbf{Remarks}\\
Candidates will be expected to know and be able to use, but not to derive, formulae for estimators and variances, along with an informal understanding of standard errors. Use of the finite population correction is not required.
Use of other practical methods of sampling: systematic sampling, cluster sampling, quota sampling, stratified random sampling and multi-stage sampling.
No formulae are required.
%=======================================================%
\subsection{Exploratory analysis}
Candidates should be prepared to examine a set of data, to choose and carry out suitable methods of analysis, to answer questions from non-statistical users and to present the analysis and conclusions in the form of a short report. The techniques required may be of the simplest kind, e.g. plotting, grouping, transforming or calculating from the data. Candidates will be expected to use box-plots and other similar graphical displays.
\textbf{Remarks}\\
Particular questions may be specified, and also particular types of non-statistical user (see below). Otherwise a brief general report on the main points in the data analysis will be appropriate.


\end{document}
