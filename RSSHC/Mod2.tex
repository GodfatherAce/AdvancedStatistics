%--------------------------------------------------------------%
\subsection*{Probability}
\begin{itemize}
\item Definitions of probability: equally likely outcomes; relative frequency; degrees of belief. 
\item Addition and multiplication of probabilities, conditional probability, statistical independence. 
\item Bayes' theorem.
\end{itemize}
%--------------------------------------------------------------%
\subsection*{Distributions}

\begin{itemize}
\item Random variables. 
\item Discrete and continuous probability distributions. 
\item Probability mass function, probability density function, cumulative distribution function. 
\item Simple theory of elementary probability distributions, including discrete uniform, Bernoulli, binomial, Poisson, geometric, negative binomial, hypergeometric, Normal, exponential, gamma and continuous uniform.
\end{itemize}
%--------------------------------------------------------------%
\subsection*{Properties of distributions}
Expectation and variance; their general properties and values for standard distributions.
Derivation of the expected value and variance of random variables with the distributions listed above. Questions may be set for other simple distributions.
Distributions, means and variances of sums of independent and identically distributed random variables and simple functions, such as aX + b. Linear combinations of independent Normally distributed variables.
Results for distributions of sums of Poisson, Normal and exponential random variables should be known.

\textbf{Remarks}
Statement and use of central limit theorem for independent, identically distributed random variables with finite variance.
Use of Normal approximations, including those for binomial and Poisson distributions.

%--------------------------------------------------------------%
\end{document}




MODULE 2: Analysis and Presentation of Data








Part 1a:

Use of rough checks for order of magnitude and leading digits in results.


Part 1b:

Approximation, limits of accuracy, rounding and accuracy of recording. 

Percentages, ratios, rates and linear interpolation. 

Distinction between discrete and continuous data.


Part 1c:

Construction and uses of frequency tables for one or more variables and contingency tables. 

Tables for presenting collections of results together with summary tables of frequencies, relevant averages, standard deviations, etc.



 









Part 5:






Calculation of least squares regression line and its interpretation. 

Correlation as a measure of linear association between two variables. 

Product-moment correlation coefficient. 

Spearman's rank correlation coefficient.
 









Part 6a:

Simple moving averages for detecting trends and for smoothing time series. 

Seasonal data.


Part 6b:

Knowledge of weighted forms of moving average.



 


