Model Thinking



Models of tipping points.
The wisdom of crowds.
Why some countries are rich and some are poor
Strategic decisions of firm and politicians.





Model of Tipping Points


In sociology, a tipping point is the event of a previously rare phenomenon becoming rapidly and dramatically more common. The phrase was coined in its sociological use by Morton Grodzins, by analogy with the fact in physics that adding a small amount of weight to a balanced object can cause it to suddenly and completely topple.


White Flight: Grodzins studied integrating American neighborhoods in the early 1960s. He discovered that most of the white families remained in the neighborhood as long as the comparative number of black families remained very small. But, at a certain point, when "one too many" black families arrived, the remaining white families would move out en masse in a process known as white flight. He called that moment the "tipping point". 





The hundredth monkey effect is a supposed phenomenon in which a learned behavior spreads instantaneously from one group of monkeys to all related monkeys once a critical number is reached. 


By generalization it means the instantaneous, paranormal spreading of an idea or ability to the remainder of a population once a certain portion of that population has heard of the new idea or learned the new ability. 


The story behind this supposed phenomenon originated with Lawrence Blair and Lyall Watson in the mid-to-late 1970s, who claimed that it was the observation of Japanese scientists. 


One of the primary factors in the promulgation of the story is that many authors quote secondary, tertiary or post-tertiary sources who have themselves misrepresented the original observations.

