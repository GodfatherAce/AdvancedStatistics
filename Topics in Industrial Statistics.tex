 
\section*{Topics in Industrial Statistics}
\begin{itemize}
\item Reliaility Engineering
\item Hazard Function
\item Reliability Function
\item Mean Time To Failure (MTTF)
\end{itemize} 

\subsection8{The Operating Characteristic (OC) Curve}
The OC Curve is used in sampling inspection. It plots the probability of accepting a batch of items against the quality level of the batch.
 
The figure below shows an 'OC' (Operating Characteristic) Curve for a sample of 50 items taken from a batch of 2000 and using a critical acceptance number 'c' of 2 (the batch will be accepted if there are two or less defectives in the sample).
 
From the curve you can see that there is about a 23% probability of accepting a batch that contains 8% of defective items.

When designing a sampling plan it is usual to decide on two points, the AQL and LQL and the associated Producer's Risk and Consumer's Risk. The necessary sample size and acceptance number for the curve to pass through these points is then calculated and hence the shape of the curve.
 
OC Curves are mainly associated with sampling inspection but they are also used to find the Average Run Length in control charts.

 
 
