
\documentclass[12pt]{article} % use larger type; default would be 10pt


\usepackage{graphicx}
\usepackage{framed}


%------------------------------------------------------%
% http://www.psychwiki.com/wiki/Analyzing_Data
% http://www.upa.pdx.edu/IOA/newsom/semclass/ho_missing.pdf
% http://www.uvm.edu/~dhowell/StatPages/More_Stuff/Missing_Data/Missing.html
% http://www.uvm.edu/~dhowell/StatPages/More_Stuff/Missing_Data/Missing.html
% http://www.stat.columbia.edu/~gelman/arm/missing.pdf
% https://onlinecourses.science.psu.edu/stat505/node/78
% http://rer.sagepub.com/content/45/4/543.full.pdf
% http://www.kdnuggets.com/faq/precision-recall.html

%------------------------------------------------------%
\title{Advanced Data Modeling - Week 12}
\author{Kevin O'Brien}

\begin{document}
\maketitle
\tableofcontents
\newpage
%------------------------------------------------------%

\newpage

\section{Missing Data}
Missing data is a common problem in all kinds of research. The way you deal with it depends on how
much data is missing, the kind of missing data (single items, a full questionnaire, a measurement
wave), and why it is missing, i.e. the reasons that the data are missing. Handling missing data is an
important step in several phases of a scientific study.

\subsection{Why do missing values occur?}

Missing values are either random or non-random.

Random missing values may occur because the subject inadvertently did not answer some questions.  - The study may be overly complex and/or long, or the subject may be tired and/or not paying attention, and miss the question. Random missing values may also occur through data entry mistakes.

Non-random missing values may occur because the subject purposefully did not answer some questions.  - The question may be confusing, so many subjects do not answer the question. Also, the question may not provide appropriate answer choices, such as \textbf{\textit{no opinion}} or \textit{\textbf{not applicable}}, so the subject chooses not to answer the question.

Also, subjects may be reluctant to answer some questions because of social desirability concerns about the content of the question, such as questions about sensitive topics like Criminal record, addiction issues, personal prejudices etc.

Missing values means reduced sample size and loss of data. You conduct research to measure empirical reality so missing values thwart the very purpose of research. The less data collected, the less data that can be analyzed, and reducing the data that can be analyzed reduces statistical power, which is the ability to detect real relationships in the data. Missing values may also indicate bias in the data. If the missing values are non-random, then the study is not accurately measuring the intended constructs. The results of your study may have been different if the missing data was not missing.
\subsection{Types of Missing Data}

\begin{itemize}
\item Missing At Random
\item Missing Completely At Random
\item Missing Not At Random
\end{itemize}



%http://missingdata.lshtm.ac.uk/
\subsubsection{Missing Completely At Random}

There are several reasons why data may be missing. They may be missing because equipment malfunctioned, the weather was terrible, people got sick, or the data were not entered correctly. Here the data are missing completely at random (MCAR). When we say that data are missing completely at random, we mean that the probability that an observation ($X_i$) is missing is unrelated to the value of $X_i$ or to the value of any other variables. Thus data on family income would not be considered MCAR if people with low incomes were less likely to report their family income than people with higher incomes.

However if a participant's data were missing because he was stopped for a traffic violation and missed the data collection session, his data would presumably be missing completely at random. Another way to think of MCAR is to note that in that case any piece of data is just as likely to be missing as any other piece of data.



%Suppose the probability of an observation being missing does not depend on observed or unobserved measurements.
%
%Then we say that the observation is Missing Completely At Random, which is often abbreviated to MCAR.
%
%(Note that in a sample survey setting MCAR is sometimes called uniform non-response.)
%
%If data are MCAR, then consistent results with missing data can be obtained by performing the analyses we would have used had their been no missing data, although there will generally be some loss of information. In practice this means that, under MCAR, the analysis of only those units with complete data gives valid inferences.
%
%An example of a MCAR mechanism would be that a laboratory sample is dropped, so the resulting observation is missing.
%
%However, many mechanisms that initially seem to be MCAR may turn out not to be.
%
%For example, a patient in a clinical trial may be lost to follow up after 'falling' under a bus; however if it is a psychiatric trial, this may be an indication of poor response to treatment. Likewise, if a response to a postal questionnaire is missing because the questionnaire was lost or stolen in the post, this may not be
% random but rather reflect the area in which the sorting office is located.

\subsubsection{Missing At Random}
Often data are not missing completely at random, but they may be classifiable as missing at random (MAR). (MAR is not really a good name for this condition because most people would take it to be synonymous with MCAR, which it is not. However, the name has stuck.)

For data to be missing completely at random, the probability that $X_i$ is missing is unrelated to the value of  $X_i$ or other variables in the analysis. But the data can be considered as missing at random if the data meet the requirement that missing-ness does not depend on the value of $X_i$ after controlling for another variable.



Missing at random (MAR) occurs when the missing-ness is related to a particular variable, but it is not related to the value of the variable that has missing data.

For example, suppose people who suffer from depression might be less inclined to report their income, and thus the rate of reported income will be related to depression. However they would report answers to other questions to the same extent that everyone else does. For those who do suffer from depression - the probability of not reporting is unrelated to their actual incomes.

As a second example, the MAR assumption would be satisfied if the probability of missing data on income depended on a person’s marital status, but within each marital status category, the probability of missing income was unrelated to
income.

%Further to the definition of MAR data, consider this: depressed people probably don't have a different income level in general. This is a supposition, and may be wrong. Let us suppose that they do have a different income level. We have a high rate of missing data among depressed individuals, the existing mean income might be significantly than it would be without missing data.


%\begin{figure}
  % Requires \usepackage{graphicx}
 % \includegraphics[scale=0.7]{Missing1.png}\\
  %\caption{Missing At Random}\label{Missing1}
%\end{figure}

However, if, within depressed patients the probability of reported income was unrelated to income level, then the data would be considered MAR, though not MCAR. Another way of saying this is to say that to the extent that missingness is correlated with other variables that are included in the analysis, the data are MAR.

%After considering MCAR, a second question naturally arises. That is, what are the most general conditions under which a valid analysis can be done using only the observed data, and no information about the missing value mechanism, Pr(r | yo, ym)?
%
%The answer to this is when, given the observed data, the missingness mechanism does not depend on the unobserved data. This is termed Missing At Random, abbreviated MAR.
%This is equivalent to saying that the behaviour of two units who share observed values have the same statistical behaviour on the other observations, whether observed or not.


\subsubsection{Missing Not at Random}

If data are not MCAR or MAR then they are classed as Missing Not at Random (MNAR). 
MNAR data is data that is missing for a specific reason (ie. the value of the variable that's missing is related to the reason it's missing)

For example, if we are studying mental health and people who have been diagnosed as depressed are less likely than others to report their mental status, the data are not missing at random. Clearly the mean mental status score for the available data will not be an unbiased estimate of the mean that we would have obtained with complete data. The same thing happens when people with low income are less likely to report their income on a data collection form.

When we have data that are MNAR we have a problem. The only way to obtain an unbiased estimate of parameters is to model missingness. In other words we would need to write a model that accounts for the missing data. That model could then be incorporated into a more complex model for estimating missing values. 

\newpage
%\begin{figure}
  % Requires \usepackage{graphicx}
 % \includegraphics[scale=0.7]{Missing2.png}\\
  %\caption{Summary of Types of Missing Data}\label{Missing2}
%\end{figure}
\subsection{Dealing with Missing Data}
There are three approaches to dealing with missing data.
\begin{itemize}
\item Option 1 Continue With the Incomplete Data
\item Option 2 Casewise deletion
\item Option 3 Imputation
\end{itemize}

%---------------------------------------------------------------%
\subsection{Option 1 : Continue With the Incomplete Data}
The first option is to leave the data as is, with the missing values in place. This is the most frequent approach, for a few reasons. First, the number of missing values are typically small. Second, missing values are typically non-random. Third, even if there are a few missing values on individual items, you typically create composites of the items by averaging them together into one new variable, and this composite variable will not have missing values because it is an average of the existing data. However, if you chose this option, you must keep in mind how SPSS will treat the missing values.

%---------------------------------------------------------------%
\subsection{Pairwise Deletions and Listwise Deletion}

\noindent \textbf{Listwise Deletion}SPSS will not include cases (subjects) that have missing values on the variable(s) under analysis. If you are only analyzing one variable, then listwise deletion is simply analyzing the existing data. If you are analyzing multiple variables, then listwise deletion removes cases (subjects) if there is a missing value on any of the variables. The disadvantage is a loss of data because you are removing all data from subjects who may have answered some of the questions, but not others (e.g., the missing data).\\ \bigskip

\noindent \textbf{PairwiseDeletion} SPSS will include all available data. Unlike listwise deletion which removes cases (subjects) that have missing values on any of the variables under analysis, pairwise deletion only removes the specific missing values from the analysis (not the entire case).

In other words, all available data is included.  - If you are conducting a correlation on multiple variables, then SPSS will conduct the bivariate correlation between all available data points, and ignore only those missing values if they exist on some variables. In this case, pairwise deletion will result in different sample sizes for each correlation. Pairwise deletion is useful when sample size is small or missing values are large because there are not many values to begin with, so why omit even more with listwise deletion.


\subsection{Option 2 : Case-Wise Deletion}
The next option is to delete cases with missing values.  - For every missing value in the dataset, you can delete the subjects with those missing values. Thus, you are left with complete data for all subjects.

The disadvantage to this approach is you reduce the sample size of your data (sometimes most of it). If you have a large dataset, then it may not be a big disadvantage because you have enough subjects even after you delete the cases with missing values.

Another disadvantage to this approach is that the subjects with missing values may be different than the subjects without missing values (i.e MNAR), so you have a non-representative sample after removing the cases with missing values.

\subsection{Option 3 : Imputation}
The last option is to replace the missing values, called imputation. There is little agreement about whether or not to conduct imputation. There is some agreement, however, in which type of imputation to conduct.  - You typically do NOT conduct \textbf{Mean substitution} or \textbf{Regression substitution}.

\begin{itemize}
\item Mean substitution is replacing the missing value with the mean of the variable.
\item Regression substitution uses regression analysis to replace the missing value. Regression analysis is designed to predict one variable based upon another variable, so it can be used to predict the missing value based upon the subject’s answer to another variable.

\end{itemize}

The favored type of imputation is replacing the missing values using different estimation methods. The \textbf{\textit{Missing Values Analysis}} add-on module in SPSS contains the estimation methods.

\subsection{Imputation}
Imputation, the practice of 'filling in' missing data with plausible values, is an attractive approach to analyzing incomplete data. It apparently solves the missing-data problem at the beginning of the analysis. However, a naive or unprincipled imputation method may create more problems than it solves, distorting estimates, standard errors and hypothesis tests, as documented by Little and Rubin (1987) and others.

For MNAR, imputation is not sufficient, because the missing data are totally different from the
available data, i.e. your complete data has become a selective group of persons. 
For MCAR and MAR, there are roughly two kinds of techniques for imputation; Single and Multiple
Imputation.

\subsection{Single Imputation}
Single imputation techniques are based on the idea that in a random sample every person can be
replaced by a new person, given that this new person is randomly chosen from the same source
population as the original person. In that case you can use the observed available data of the
other persons to make an estimation of the distribution of the test result in the source population.
It is called single imputation, because each missing is imputed once.

There are many methods for single imputation, such as replacement by the mean, regression,
and expected maximization. Expected maximization is preferred, because in the other methods
the variance and standard error are reduced and the chance for Type II errors increases. 
%--------------------------------------------------------------------------------%
\subsection{Multiple Imputation}

The difference with single imputation is that in MI the value is imputed for several times. There are
more imputed datasets created. The different imputations are then based on random draws of
different estimations of the underlying distribution in the source population. In this way, the
imputed data comes from different distributions and therefore are less look alike. There is more
uncertainty created in the dataset. Therefore the standard error increases. The amount of
imputations is dependent on the amount of missing data, but mostly 5 to 10 imputations are
enough. A drawback of this method it that several imputed datasets are created and that the
statistical analysis has to be repeated in each dataset. Finally, results have to be pooled in a
summary measure. Most statistical packages can do this automatically.
\end{document}
