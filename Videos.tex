% http://www.r-tutor.com/gpu-computing/clustering/distance-matrix

% http://www.ats.ucla.edu/stat/r/dae/exlogit.htm

% www.stat.columbia.edu/~martin/W2024/R7.pdf

% 

%----------------------------------%
\section*{Principal Components Analysis}
%----------------------------------%
\begin{frame}[fragile]
\frametitle{Principal Components Analysis}

\begin{itemize}

\item \texttt{princomp} performs a principal 
components analysis on the given numeric data 
matrix and returns the results as an object of 
class princomp.

\item \texttt{prcomp} performs a principal 
components analysis on the given data matrix and returns the results as an object of class prcomp.

\end{itemize}

\end{frame}
%----------------------------------%
\begin{frame}[fragile]
\frametitle{Principal Components Analysis}

The variances of the variables in the
USArrests data vary by orders of magnitude, so scaling is appropriate

\begin{framed}
\begin{verbatim}
(pc.cr <- princomp(USArrests))  # inappropriate
princomp(USArrests, cor = TRUE) # =^= prcomp(USArrests, scale=TRUE)

\end{verbatim}
\end{framed}

The standard deviations differ by a factor of sqrt(49/50)
\end{frame}

%----------------------------------%
\begin{frame}[fragile]
\frametitle{Principal Components Analysis}


\textbf{Screeplots}

\begin{itemize}

\item \texttt{screeplot} plots the variances against the number of the principal component. This is also the plot method for classes "princomp" and "prcomp".
\end{itemize}

\end{frame}
%----------------------------------%
\section*{Exact Logistic Regression }
%----------------------------------%
\begin{frame}[fragile]
\frametitle{Exact Logistic Regression}

\begin{itemize}

\item Exact logistic regression is used to model 
binary outcome variables in which the log odds of 
the outcome is modeled as a linear combination of 
the predictor variables. 
\item It is used when the sample size is too small for a regular logistic regression (which uses the standard maximum-likelihood-based estimator) and/or when some of the cells formed by the outcome and categorical predictor variable have no observations. 
\item The estimates given by exact logistic regression do not depend on asymptotic results.

\end{itemize}

\end{frame}

%----------------------------------%

\begin{frame}[fragile]
\frametitle{Exact Logistic Regression}

\begin{framed}
\begin{verbatim}

library(elrm)

\end{verbatim}
\end{framed}

\end{frame}

%----------------------------------%
\section*{Tukey Honest Significant Differences}
%----------------------------------%

\begin{frame}[fragile]
\frametitle{Tukey Honest Significant Differences}

\textbf{Tukey Honest Significant Differences}

\begin{itemize}

\item Create a set of confidence intervals on 
the differences between the means of the levels 
of a factor with the specified family-wise probability of coverage. 
\item The intervals are based on the Studentized range statistic, Tukey's ‘Honest Significant Difference’ method.

\end{itemize}

\end{frame}
%----------------------------------%

\begin{frame}[fragile]
\frametitle{Tukey Honest Significant Differences}


\begin{framed}
\begin{verbatim}

summary(fm1 <- aov(breaks ~ wool + tension, data = warpbreaks))
TukeyHSD(fm1, "tension", ordered = TRUE)
plot(TukeyHSD(fm1, "tension"))

\end{verbatim}
\end{framed}

\end{frame}
%----------------------------------%
\section*{Grubbs Test for Outliers}
%----------------------------------%

\begin{frame}[fragile]
\frametitle{Grubbs Test for Outliers}

\textbf{The outliers package}
\begin{framed}
\begin{verbatim}

install.packages("outliers")
library(outliers)

\end{verbatim}
\end{framed}

\end{frame}
%----------------------------------%
\section*{Cook's Distance - Linear Models}
%----------------------------------%

% Influence
% What is Cook's Distance

%----------------------------------%
\begin{frame}[fragile]
\frametitle{Cook's Distance}

\begin{framed}
\begin{verbatim}
d1 <- cooks.distance(ols)
\end{verbatim}
\end{framed}

\end{frame}
%----------------------------------%
\section*{ANOVA - Analysis of Variance}
%----------------------------------%
\begin{frame}[fragile]
\frametitle{ANOVA - Analysis of Variance}
\texttt{aov}

\begin{framed}
\begin{verbatim}

set.seed(1234)
X=rnorm(16, mean=100,sd=10)

\end{verbatim}
\end{framed}

\end{frame}
%----------------------------------%
\begin{frame}[fragile]
\frametitle{ANOVA - Analysis of Variance}

\begin{itemize}
\item Single replicate case
\item Multiple Replicate case
\end{itemize}

\end{frame}
%----------------------------------%
\begin{frame}[fragile]
\frametitle{factors}


\begin{framed}
\begin{verbatim}

trt1 <- c("Hi","Lo","Hi","Lo""Hi","Lo""Hi","Lo")

trt1 <- factor(trt1)

\end{verbatim}
\end{framed}



\end{frame}
%----------------------------------%



\end{document}
