Previously we have seen the confidence intervals for regression coefficients in linear models. These 
confidence intervals are computed using the standard error values , which are available on the 
output of the summary() command, when using R. 
 
Recall that confidence intervals are generally constructed using point estimates, quantiles and 
standard errors. 
 
 In this class we will look at two more type of intervals. 
 
 Confidence Intervals for Fitted Values 
 Prediction Intervals for Fitted Values 
 
Recall: a fitted value Y* is a estimate for the response variable, as determined by a linear model. The 
difference between the observed value and the corresponding fitted value is known as the residual. 
 
The residual standard error is the conditional standard deviation of the dependent variable Y given a 
value of the independent variable X. The calculation of this standard error follows from the 
definition of the residuals. 
 
 
The residual standard error is often called the root mean square error (RMSE), and is a measure of 
the differences between values predicted by a model or an estimator and the values actually 
observed from the thing being modelled or estimated. 
 
Since the residual standard error is a good measure of accuracy, it is ideal if it is small. 
