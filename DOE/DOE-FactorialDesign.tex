
THE PROS AND CONS OF FACTORIAL DESIGN

Factorial designs are extremely useful to psychologists and field scientists as a preliminary study, allowing them to judge whether there is a link between variables, whilst reducing the possibility of experimental error and confounding variables.

The factorial design, as well as simplifying the process and making research cheaper, allows many levels of analysis. As well as highlighting the relationships between variables, it also allows the effects of manipulating a single variable to be isolated and analyzed singly.

The main disadvantage is the difficulty of experimenting with more than two factors, or many levels. A factorial design has to be planned meticulously, as an error in one of the levels, or in the general operationalization, will jeopardize a great amount of work.

Other than these slight detractions, a factorial design is a mainstay of many scientific disciplines, delivering great results in the field.



Read more: http://www.experiment-resources.com/factorial-design.html#ixzz26hBQjzMY

