

Blocking

Factorial Designs

Full Factorial Designs

Fractional Factorial Designs

Aliasing

Interaction

Sample Question


Blocking


Blocking is a procedure under which experimental units are grouped into "blocks" that are expected to be as alike as possible within themselves but may be consistently different from each other. 


The blocks should then remove a possible element of systematic variation so that the residual mean square in the usual analysis of variance truly estimates just experimental error and is not inflated by such a source of consistent variation. Comparisons between treatment means are then more precise.


Factorial Designs


Factorial designs are the most common type of experimental design. In a factorial design several factors are controlled at two or more levels, and the effect on the response is investigated.

There are two main types: Full Factorial Designs and Fractional Factorial Designs. 


In Fractional Factorial designs the amount of testing is reduced, but the downside is that some interactions and factors are aliased. 


One of the limitations of Fractional Factorial designs is that the number of runs is 2n, if there are more than a few factors this leads to large gaps in the available options (4, 8, 16, 32, 64, 128, etc.). Plackett-Burman designs overcome this, but at the expense of confounding. Other approaches to the Design of Experiments include Response Surface Methods and Taguchi Designs.



Full Factorial Designs


Factorial designs involve testing several factors at two levels, high and low. In a Full Factorial experiment every possible combination of factors and permutations is tested. If there are n factors there are 2n treatments.



The design matrix shows the combinations for a 22 design











Treatment
 

A
 

B
 

AB
 



1
 

-1
 

-1
 

+1
 



2
 

+1
 

-1
 

-1
 



3
 

-1
 

+1
 

-1
 



4
 

+1
 

+1
 

+1
 


Fractional Factorial Designs

 

Factorial designs involve testing several (n) factors at two levels, high and low. A Fractional Factorial experiment uses only a half (2n-1), a quarter (2n-2), or some other division by a power of two of the number of treatments that would be required for a Full Factorial Experiment.


A Full Factorial design with 3 factors (23 design) would require 8 treatments. The example shows a 23-1 design, it requires only 4 treatments:












Treatment
 

A
 

B
 

C
 



1
 

-1
 

-1
 

+1
 



2
 

+1
 

-1
 

-1
 



3
 

-1
 

+1
 

-1
 



4
 

+1
 

+1
 

+1
 




The Fractional Factorial design is created by aliasing Factor C with the Interaction AB. This is a Resolution III design.







Aliasing


In experimental design when two interactions, or a main effect and an interaction, share the same column, and so cannot be individually analyzed then their effects are aliased.

In the 23-1 design the factor C is aliased with the interaction AB:











Treatment
 

A
 

B
 

AB + C
 



1
 

-1
 

-1
 

+1
 



2
 

+1
 

-1
 

-1
 



3
 

-1
 

+1
 

-1
 



4
 

+1
 

+1
 

+1
 

When the design is analyzed it is not possible to distinguish between the effects of changes in the settings of factor C or the effects of an interaction between factors A and C. Thus this design would only be useful if you believed, from other information, that interactions between A and C would not be significant.


Interaction 

In many processes the factors interact, the combined effect is not the sum of the individual effects. The figure below uses the well known danger of combining alcohol with some medications to illustrate the idea:









 


Interactions are an important consideration in experimental design.


Sample Question


Explain what is meant by the main effect of a factor and the interaction between two factors in a 22 factorial experiment.

•
The main effect of a factor in a 22 experiment is the difference between the results with the factor at its high level and those with it at its low level; thus, for factor A, it is given by ab + a – (b + (1)) 


•
[an average difference might be used, i.e. with a divisor of 2]. 


•
Similarly, for B it is given by ab + b – (a + (1)).


•
The remaining independent comparison that is possible is ab + (1) – (a + b). 


•
By rearranging this as (ab – b) – (a – (1)), it can be seen to measure the difference between the "responses" to factor A at the high level of B and those at the low level of B. 


•
Equivalently, the roles of A and B can be interchanged throughout this. It is called the interaction between A and B.



Introduction to Latin Squares

A Latin square is used in experimental designs in which one wishes to compare treatments and to control for two other known sources of variation. To use a Latin square for an experiment comparing n treatments we will need to have n levels for each of the two sources of variation for which we wish to control.


Latin squares were first used in agricultural experiments. It was recognized that within a field there would be fertility trends running both across the field and up and down the field. So in an experiment to test,say, four different fertilisers, A, B, C and D, the field would divided into four horizontal strips and four vertical strips, thus producing 16 smaller plots. A Latin square design will give a random allocation of fertiliser type to a plot in such a way that each fertiliser type is used once in each horizontal strip (row) and once in each vertical strip (column).


Example of Latin Square Design


Suppose that we want to test five drugs A;B;C;D;E for their effect in alleviating the symptoms of a chronic disease. Five patients are available for a trial, and each will be available for five weeks. Testing a single drug requires a week. Thus an experimental unit is a ‘patient-week’.


The structure of the experimental units is a rectangular grid (which happens to be square in this case); there is no structure on the set of treatments. We can use the Latin square to allocate treatments. If the rows of the square represent patients and the columns are weeks, then for example the second patient, in the third week of the trial, will be given drug D. Now each patient receives all five drugs, and in each week all five drugs are tested.


immer data set


Yields from a Barley Field Trial

Description


The immer data frame has 30 rows and 4 columns. 

Five varieties of barley were grown in six locations in each of 1931 and 1932.


The variety of barley ("manchuria", "svansota", "velvet", "trebi" and "peatland").

Y1:    Yield in 1931.

Y2:    Yield in 1932.


Taguchi


http://www.statsoft.com/textbook/experimental-design/#taguchi


