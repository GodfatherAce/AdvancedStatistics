Proportional hazards models are a class of survival models in statistics. 


Survival models relate the time that passes before some event occurs to one or more covariates that may be associated with 
that quantity of time. In a proportional hazards model, the unique effect of a unit increase in a covariate is multiplicative 
with respect to the hazard rate. 

This is in contrast to additive hazard models, wherein a treatment may increase ones hazard rate by a fixed amount independently of other
covariates.

For example, taking a drug may halve one's hazard rate for a stroke occurring, or, changing
the material from which a manufactured component is constructed may double its hazard rate for failure. 

Other types of survival models such as accelerated failure time models do not exhibit proportional hazards. 
The accelerated failure time model describes a situation where 
the biological or mechanical life history of an event is accelerated.

%===============================================%
\subsubsection*{Composition}
Consider survival models to consist of two parts
\begin{enumerate}
\item The Hazard Function
\item The Effects Parameter
\end{enumerate}
