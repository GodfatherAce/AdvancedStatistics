
Interval estimates for fitted values

The fitted value

ˆ Y for a value x of the explanatory variable is the best estimate of

the mean of the response variable

Y for that value x. It is given by

ˆ

Y = A + Bx.



http://www.weibull.com/DOEWeb/confidence_intervals_in_multiple_linear_regression.htm



Simple Linear Regression as used in Chemistry

Limits of Detection

detection of analyte

Blank Signal


Method Comparison Studies
•
Simple Linear Regression

•
Deming Regression

•
Orthogonal Regression

•
Bland Altman Plot




Alternatives to SLR
•
Polynomial Regression

•
Weighted Linear Regression

•
Comparing Candidate Models using AIC 

•
Likelihood ratio testing



Root Mean Square Error

The root mean square error (RMSE)) is a measure of the differences between values predicted by a model or an estimator and the values actually observed from the thing being modeled or estimated.


Since the RMSE is a good measure of accuracy, it is ideal if it is small.













http://www-stat.wharton.upenn.edu/~stine/stat621/lecture3.621.pdf


“Statistical extrapolation penalty”

CI for regression line grows wider as get farther away from the

mean of the predictor.

 

Regression ANOVA

Sums of Squares Identities

F test

( Remark : we rearranged the sequence of material so that we would have covered important material in lectures before applying this material in Labs)


Variable Selection Procedures

Stepwise Regression

Forward Selection

Backward Elimination

stepAIC( )

model selection


http://www.chem.utoronto.ca/coursenotes/analsci/StatsTutorial/ErrRegr.html

http://www.medcalc.org/manual/deming_regression.php



http://www.r-tutor.com/elementary-statistics/simple-linear-regression/prediction-interval-linear-regression


A few remarks relating to upcoming classes  :


I advise a brief revision of the Normal Distribution as it is going to a fundamental part of  Statistical Process Control. A part of the course thar we are going to be studying soon.


In this part of the course I will also be covering Tolerance Intervals. We will be revisiting Confidence Intervals and Prediction Intervals also. and highlighting the differences between all three.


The classes will not be going ahead next Friday as it is one of ULs open days.

So as not to go out of sync, the lab schefuled for next monday will concentrate on a few special topics related to using R.


One key topic in mind for that class is using R Packages. Previously we have discussed  the nortest package, required to implement the Anderson Darling test for normality. There are in fact several thousand more, covering a wide variety of topics. A directory of theseis maintained at the Comprehensive R Archive Network (CRAN) website.


Some packages I intend on using before the end of the course are qcc spc and MethComp.


Another major area of the course is Experimental Design.


The mid term exam will take place on Week 8 in the computer laboratory.  Precise dates will be confirmed shortly. Please attend the session as instructed by you timetable.


Past papers for the midterm exam are available on SULIS. Please note that your own exam will deal specifically with Inference Procedures ( i.e. hypothesis tests and confidence intervals  ) and linear models. Also you exam is worth 20% of the overall grade, rather than 15% for that module.






