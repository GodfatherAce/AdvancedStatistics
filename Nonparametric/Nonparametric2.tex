Most of the statistical methods described in this book are called parametric methods. The focal point of
parametric analysis is some population parameter for which the sampling statistic follows a known distribution,
with measurements being made at the interval or ratio scale. When one or more of these requirements or
assumptions are not satisfied, then the so-called nonparametric methods can be used. An alternative term is
distribution-free methods, which focuses particularly on the fact that the distribution of the sampling statistic is
not known.
If use of a parametric test, such as the t test, is warranted, then we would always prefer its use to the
nonparametric equivalent. This is so because if we use the same level of significance for both tests, then the
power associated with the nonparametric test is always less than the parametric equivalent. (Recall from
Section 10.4 that the power of a statistical test is the probability of rejecting a false null hypothesis.)
Nonparametric tests often are used in conjunction with small samples, because for such samples the central
limit theorem cannot be invoked (see Section 8.4).
Nonparametric tests can be directed toward hypotheses concerning the form, dispersion, or location
(median) of the population. In the majority of the applications, the hypotheses are concerned with the value of a
median, the difference between medians, or the differences among several medians. This contrasts with the
parametric procedures that are focused principally on population means.
Of the statistical tests already described in this book, the chi-square test covered in Chapter 12 is a
nonparametric test. Recall, for example, that the data that are analyzed are at the nominal scale (categorical
data). A separate chapter was devoted to the chi-square test because of the extent of its use and the variety of its
applications.
