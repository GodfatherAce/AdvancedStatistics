THE RUNS TEST FOR RANDOMNESS
Where a run is a series of like observations, the runs test is used to test the randomness of a series of
observations when each observation can be assigned to one of two categories.
EXAMPLE 5. For a random sample of n ¼ 10 individuals, suppose that when they are categorized by sex, the sequence of
observations is: M, M, M, M, F, F, F, F, M, M. For these data there are three runs, or series of like items.
For numeric data, one way by which the required two-category scheme can be achieved is to classify each
observation as being either above or below the median of the group. In general, either too few runs or too many
runs than would be expected by chance would result in rejecting the null hypothesis that the sequence of
observations is a random sequence.
The number of runs of like items is determined for the sample data, with the symbol R used to designate the
number of observed runs. Where n1 equals the number of sampled items of one type and n2 equals the number of
sampled items of the second type, the mean and the standard error associated with the sampling distribution of
the R test statistic when the sequence is random are


\[ Equation 1 \]

\[ Equation 2\]
