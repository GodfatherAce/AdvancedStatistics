

\section*{Robust Statistics}

Robust statistics provides an alternative approach to standard statistical methods, 
such as those for estimating location, scale and regression parameters. 
The motivation is to produce estimators that are not unduly affected by small departures 
from the model assumptions under which these standard methods are usually 

derived: the standard methods are comparatively badly affected.

Examples of robust and non-robust statistics

\begin{itemize}
\item The median is a robust measure of central tendency, while the mean is not; for instance, the median has a breakdown point of 50%, while the mean has a breakdown point of 0% (a single large sample can throw it off).
\item The median absolute deviation and interquartile range are robust measures of statistical dispersion, while the standard deviation and range are not.
\end{itemize}

\subsection*{Estimators}
Trimmed estimators and Winsorised estimators are general methods to make statistics more robust. M-estimators are a general class of robust statistics.






