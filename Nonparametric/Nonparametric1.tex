Nonparametric statistics are statistics not based on parameterized families of probability distributions. They include both descriptive and inferential statistics. The typical parameters are the mean, variance, etc. Unlike parametric statistics, nonparametric statistics make no assumptions about the probability distributions of the variables being assessed. The difference between parametric model and non-parametric model is that the former has a fixed number of parameters, while the latter grows the number of parameters with the amount of training data.[1] Note that the non-parametric model is not none-parametric.


%=====================================================%

Nonparametric
Statistics
17.1 SCALES OF MEASUREMENT
Before considering how nonparametric methods of statistics differ from the parametric procedures that
constitute most of this book, it is useful to define four types of measurement scales in terms of the precision
represented by reported values.
In the nominal scale, numbers are used only to identify categories. They do not represent any amount or
quantity as such.
EXAMPLE 1. If four sales regions are numbered 1 through 4 as general identification numbers only, then the nominal
scale is involved, since the numbers simply serve as category names.
In the ordinal scale, the numbers represent ranks. The numbers indicate relative magnitude, but the
differences between the ranks are not assumed to be equal.
EXAMPLE 2. An investment analyst ranks five stocks from 1 to 5 in terms of appreciation potential. The difference in the
appreciation potential between the stocks ranked 1 and 2 generally would not be the same as, say, the difference between the
stocks ranked 3 and 4.
In the interval scale, measured differences between values are represented. However, the zero point is
arbitrary, and is not an “absolute” zero. Therefore, the numbers cannot be compared by ratios.
EXAMPLE 3. In either the Fahrenheit or the Celsius temperature scales, a 58 difference, from, say, 708F, to 758F is the
same amount of difference in temperature as from 808F to 858F. However, we cannot say that 608F is twice as warm as 308F,
because the 08F point is not an absolute zero point (the complete absence of all heat).

%=====================================================%
