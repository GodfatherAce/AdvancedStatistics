SEVERAL INDEPENDENT SAMPLES: THE KRUSKAL–WALLIS TEST
The Kruskal–Wallis test is used to test the null hypothesis that several populations have the same
medians. As such, it is the nonparametric equivalent of the one-factor completely randomized design of the
322 NONPARAMETRIC STATISTICS [CHAP. 17
analysis of variance. It is assumed that the several populations have the same form and dispersion for the
above hypothesis to be applicable, because differences in form or dispersion would also lead to rejection of
the null hypothesis. The values for the several independent random samples are required to be at least at the
ordinal scale.
The several samples are first viewed as one array of values, and each value in this combined group is
ranked from lowest to highest. For equal values the mean rank is assigned to the tied values. If the null
hypothesis is true, the average of the ranks for each sample group should be about equal. The test statistic
calculated is designated H and is based on the sum of the ranks in each of the several random samples, as
follows:

where N ¼ combined sample size of the several samples
(note that N does not designate population size in this case)
Rj ¼ sum of the ranks for the jth sample or treatment group
nj ¼ number of observations in the jth sample
Given that the size of each sample group is at least nj " 5 and the null hypothesis is true, the sampling
distribution of H is approximately distributed as the x
2 distribution with df ¼ K ! 1, where K is the number
of treatment or sample groups. The x
2 value that approximates the critical value of the test statistic is
always the upper-tail value. This test procedure is analogous to the upper tail of the F distribution being
used in the analysis of variance.
For tied ranks the test statistic H should be corrected. The corrected value of the test statistic is designated
Hc and is computed as follows:
