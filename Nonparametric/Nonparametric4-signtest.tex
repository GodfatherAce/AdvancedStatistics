ONE SAMPLE: THE SIGN TEST
The sign test can be used to test a null hypothesis concerning the value of the population median. Therefore,
it is the nonparametric equivalent to testing a hypothesis concerning the value of the population mean. The
values in the random sample are required to be at least at the ordinal scale, with no assumptions required about
the form of the population distribution.
The null and alternative hypotheses can designate either a two-sided or a one-sided test. Where Med
denotes the population median and Med0 designates the hypothesized value, the null and alternative hypotheses
for a two-sided test are
H0: Med ¼ Med0
H1: Med = Med0
A plus sign is assigned for each observed sample value that is larger than the hypothesized value of the
median, and a minus sign is assigned for each value that is smaller than the hypothesized value of the median. If
a sample value is exactly equal to the hypothesized median, no sign is recorded and the effective sample size is
thereby reduced. If the null hypothesis regarding the value of the median is true, the number of plus signs should
approximately equal the number of minus signs. Or put another way, the proportion of plus signs (or minus
signs) should be about 0.50. Therefore, the null hypothesis tested for a two-sided test is H0:p ¼ 0:50, wherep is
the population proportion of the plus (or the minus) signs. Thus, a hypothesis concerning the value of the
median in fact is tested as a hypothesis concerning p. If the sample is large, the normal distribution can be used,
as described in Section 11.4.
See Problem 17.2 for the use of the sign test to test a null hypothesis concerning the population median.
