

A3.	  


What is autocorrelation?Discuss the consequences of autocorrelation for OLS estimation.(30%)


Derive the Durbin-Watson test statistic and describe how it is used to test for autocorrelation. What are the weaknesses of this test? (35%)


Describe a single method that you would use to correct for autocorrelation if the autocorrelation coefficient ρ is not known.                                                                                           (35%)











Definition: The Durbin-Watson test is a test for first-order serial correlation in the residuals of a time series regression. A value of 2.0 for the Durbin-Watson statistic indicates that there is no serial correlation.


This result is biased toward the finding that there is no serial correlation if lagged values of the regressors are in the regression. 


Formally, the statistic is:
If e_t is the residual associated with the observation at time t, then the test statistic is

\[{\displaystyle d={\sum _{t=2}^{T}(e_{t}-e_{t-1})^{2} \over {\sum _{t=1}^{T}e_{t}^{2}}},} d={\sum _{{t=2}}^{T}(e_{t}-e_{{t-1}})^{2} \over {\sum _{{t=1}}^{T}e_{t}^{2}}},\]
where T is the number of observations.
where the series of e_t are the residuals from a regression.


%=============================%
https://en.wikipedia.org/wiki/Durbin%E2%80%93Watson_statistic

To test for positive autocorrelation at significance α, the test statistic d is compared to lower and upper critical values (dL,α and dU,α):

If d < dL,α, there is statistical evidence that the error terms are positively autocorrelated.
If d > dU,α, there is no statistical evidence that the error terms are positively autocorrelated.
If dL,α < d < dU,α, the test is inconclusive.
Positive serial correlation is serial correlation in which a positive error for one observation increases the chances of a positive error for another observation.

To test for negative autocorrelation at significance α, the test statistic (4 − d) is compared to lower and upper critical values (dL,α and dU,α):

If (4 − d) < dL,α, there is statistical evidence that the error terms are negatively autocorrelated.
If (4 − d) > dU,α, there is no statistical evidence that the error terms are negatively autocorrelated.
If dL,α < (4 − d) < dU,α, the test is inconclusive.
