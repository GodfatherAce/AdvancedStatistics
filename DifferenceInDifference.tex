Difference in differences


%================================================================================================%


Difference in differences (sometimes 'Difference-in-Differences',[1] 'DID',[2] or 'DD'[3]) is a statistical technique used in econometrics and quantitative sociology, which attempts to mimic an experimental research design using observational study data. It calculates the effect of a treatment (i.e., an explanatory variable or an independent variable) on an outcome (i.e., a response variable or dependent variable) by comparing the average change over time in the outcome variable for the treatment group to the average change over time for the control group. This method may be subject to certain biases (mean reversion bias, etc.), although it is intended to eliminate some of the effect of selection bias. In contrast to a within-subjects estimate of the treatment effect (which measures differences over time) or a between-subjects estimate of the treatment effect (which measures the difference between the treatment and control groups), the DID measures the difference in the differences between the treatment and control group over time.
 



Contents
  [hide]  1 General Definition
 2 Formal Definition
 3 Assumptions
 4 Implementation
 5 Card & Krueger (1994) example
 6 Critics
 7 See also
 8 References
 9 Further reading
 10 External links
 

General Definition[edit]
 





Difference in differences requires data measured at two or more different time periods. In the example pictured, the treatment group is represented by the line P and the control group is represented by the line S. Both groups are measured on the outcome (dependent) variable at Time 1 before either group has received the treatment (i.e., the independent or explanatory variable), represented by the points P1 and S1. The treatment group then receives or experiences the treatment and both groups are again measured after this at Time 2. Not all of the difference between the treatment and control groups at Time 2 (that is, the difference between P2 and S2) can be explained as being an effect of the treatment, because the treatment group and control group did not start out at the same point at Time 1. DID therefore calculates the "normal" difference in the outcome variable between the two groups (the difference that would still exist if neither group experienced the treatment), represented by the dotted line Q. (Notice that the slope from P1 to Q is the same as the slop from S1 to S2.) The treatment effect is the difference between the observed outcome and the "normal" outcome (the difference between P2 and Q).
 
 
%================================================================================================%



Formal Definition[edit]
 
Consider the model
 


where is the dependent variable for individual , given and . The dimensions and may for example be state and time. and is then the vertical intercept for and respectively. is a dummy variable indicating treatment status, is the treatment effect, and is an error term.
 
Let
 
,
 
,
 
,
 
,
 
,
 
and suppose for simplicity that and . Then
 




.
 
The strict exogeneity assumption then implies that
 
.
 
Without loss of generality, assume that and , giving the DID estimator
 
,
 
which can be interpreted as the treatment effect of the treatment indicated by .
 
Assumptions[edit]
 




Illustration of the parallel trend assumption 
All the assumptions of the OLS model apply equally to DID. In addition, DID requires a parallel trend assumption. The parallel trend assumption says that are the same in both and . Given that the formal definition above accurately represents reality, this assumption automatically holds. However, a model with may well be more realistic.
 
As illustrated to the right, the treatment effect is the difference between the observed value of y and what the value of y would have been with parallel trends, had there been no treatment. The Achilles' heel of DID is when something other than the treatment changes in one group but not the other at the same time as the treatment, implying a violation of the parallel trend assumption.
 
To guarantee the accuracy of the DID estimate, the composition of individuals of the two groups is assumed to remain unchanged over time. When using a DID model, various issues that may compromise the results, such as autocorrelation and Ashenfelter dips, must be considered and dealt with.
