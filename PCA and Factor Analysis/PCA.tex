\section{Principal Component Analysis}
PCA is defined as the orthogonal linear transformation that transforms data to a new co-ordinate system, such that the greatest variance by any projection
of the projection of the data comes to lie on the first co-ordinate (called the first principal component).

\[ Y^{T} = X^{T}W = V \Sigma \]

where $V \Sigma W^{T}$ is the singular value decomposition (SVD) of the data matrix $X^{T}$

%==========================================================================%

This is a Technique used to reduce multidimensional data sets to lower dimensions for analysis.
PCA is also known as KArhunen Loeve Transform, Hotelling Transform, and Proper Orthogonal Decomposition.

It is mostly used as tool in Exploratory Data Analysis and used for predictive models.

PCA involves calculation of Eigenvalue Deomposition/ Single Value Decomposition for a data set, usually mean centering the data for each attribute.

It is mathematically defined as an orthogonal linear transformation that transforms the data to a new co-ordinate system, such that the greatest variance
by any projection of the data comes to lie on the first co-ordinate (called the first principal component) the second greates variance ont eh second princiiapl component anf so on. 

%==========================================================================%
