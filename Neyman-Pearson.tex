 
RSS Graduate Diploma Paper 2 – Statistical Inference
 
 

Explain the Neyman-Pearson approach to hypothesis testing when the null and alternative hypotheses are simple.
 
Suppose that θ is the parameter of a distribution. We want to test
 

 
where 0 and 1(0) are given
 
Let α be the required significance level. The Neyman-Pearson approach is to choose
the test with the largest power at 1 , subject to its size being ≤α .
 
The Neyman-Pearson lemma shows that this property is satisfied by a likelihood ratio test.


Decision making: the actions "accept
H0" or "accept H1" must be taken after
analysing the data. Usually no wider issues are involved; the data are relevant only to
the immediate situation (e.g. quality control – either want to stop the production line
or let it continue).
 
Strength of evidence: it is not necessarily expected that the current experiment will
lead to immediate actions, rather that it will add to previously gained information.
Wider issues are involved and it is often felt important that significant evidence is
found in several independent studies (e.g. at independent centres). In principle,
p values can be combined (meta-analysis).
 
One application is clinical trials.The contrast should not be taken too far. In the former case, a value near the critical value ("just accept" H0 or "just accept" H1) may lead to a suspension of action until
further evidence is obtained. On the other hand, a "very significant" result in the
second case may lead to immediate action.
 
Significance level: in decision making, this will deliberately be chosen to reflect the
"cost" of wrongly rejecting H0 (e.g. stopping the production line when nothing is
wrong). In the strength of evidence approach, it is customary to use one of the
traditional values (e.g. 0.05), or to quote the exact p-value.
 
Sample size: in decision making, this will be deliberately chosen to reflect the cost of
making wrong decisions (e.g. continuing operating the production line when in fact
there is a fault). In the strength of evidence approach, it is common practice to ensure
that the sample size is sufficiently large that the power of detecting an effect of
practical importance is sufficiently high.
