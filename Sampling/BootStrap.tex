\documentclass[serif]{beamer} % Get Computer Modern math font.
\hypersetup{colorlinks,linkcolor=,urlcolor=red}

%  To create handout using article mode: Comment above and uncomment below (2 places)
%\documentclass[12pt]{article}
%\usepackage{beamerarticle}
%\usepackage[colorlinks=true, pdfstartview=FitV, linkcolor=blue, citecolor=blue, urlcolor=red]{hyperref} % For live Web links with href in article mode
%\usepackage{amsmath} % For \binom{n}{y}
%\usepackage{graphicx} % To include pdf files!
%\usepackage{fullpage}


\usefonttheme{serif} % Looks like Computer Modern for non-math text -- nice!
\setbeamertemplate{navigation symbols}{} % Suppress navigation symbols
% \usetheme{Berlin} % Displays sections on top
\usetheme{Frankfurt}  % Displays section titles on top: Fairly thin but still swallows some material at bottom of crowded slides
%\usetheme{Berkeley}

\usepackage[english]{babel}
\usepackage{amsmath} % for binom
% \usepackage{graphicx} % To include pdf files!
% \definecolor{links}{HTML}{2A1B81}
% \definecolor{links}{red}
\setbeamertemplate{footline}[frame number] 

\mode<presentation>
% \mode<handout>{\setbeamercolor{background canvas}{bg=black!5}}

% Comment this out for handout
\title{The Bootstrap\footnote{See last slide for copyright information.}}
\subtitle{STA442/2101 Fall 2012}
\date{} % To suppress date

\begin{document}

% More material required for handout in article mode. Also eliminate vspace
%\title{The Multinomial Model\footnote{See last slide for copyright information.}}
%\subtitle{STA 442/2101: Fall 2012}
%\date{} % To suppress date
%\maketitle

% Don't want a separate title page for handout
\begin{frame}
  \titlepage
\end{frame}

\begin{frame}
\frametitle{Background Reading}
%\framesubtitle{It may be a little bit helpful.} 
  \begin{itemize}
    \item Davison's \emph{Statistical models} has almost nothing.
    \item The White whale has some material that is specific to regression, and is best left for later.
    \item The best we can do for now is the \href{http://en.wikipedia.org}{Wikipedia}
          under 
         \href{http://en.wikipedia.org/wiki/Bootstrapping_(statistics)}
         {Bootstrapping (Statistics)}
  \end{itemize}
\end{frame}

\begin{frame}
\frametitle{Overview}
\tableofcontents
\end{frame}

\section{Sampling distributions}

\begin{frame}
\frametitle{Sampling distributions} 
%\framesubtitle{} 
  \begin{itemize}
    \item Let $\mathbf{X} = (X_1, \ldots, X_n)$ be a random sample from some distribution $F$.
    \item $T=T(\mathbf{X})$ is a statistic (could be a vector of statistics).
    \item Need to know about the distribution of $T$.
    \item Sometimes it's not easy, even asymptotically.
  \end{itemize}
\end{frame}

\begin{frame}
\frametitle{Sampling distribution of $T$: The elementary version}
%\framesubtitle{} 
  \begin{itemize}
    \item Sample repeatedly from this population (pretend).
    \item For each sample, calculate $T$.
    \item Make a relative frequency histogram of the $T$ values you observe.
    \item As the number of samples becomes very large, the histogram approximates the distribution of $T$.
  \end{itemize}
\end{frame}

\section{Bootstrap}

\begin{frame}
\frametitle{What is a bootstrap?}
\framesubtitle{Pull yourself up by your bootstraps} 

\begin{center}
\includegraphics[width=2.5in]{Dr_Martens,_black,_old.jpg}
\end{center}

{\scriptsize
This photograph was taken by Tarquin. It is licensed under a 
\href{http://creativecommons.org/licenses/by-sa/3.0/deed.en_US}
     {Creative Commons Attribution - ShareAlike 3.0 Unported License}.
For more information, see the entry at the
\href{http://commons.wikimedia.org/wiki/File:Dr_Martens,_black,_old.jpg}
     {wikimedia site}.
} % End size
\end{frame}

\begin{frame}
\frametitle{The (statistical) Bootstrap}
%\framesubtitle{Bradley Efron, 1979} 
  \begin{itemize}
    \item Select a random sample from the population.
    \item If the sample size is large, the sample is similar to the population.
    \item Sample repeatedly from the sample. This is called \emph{resampling}.
    \item Sample from the sample? Think of putting the sample data values in a jar \ldots
    \item Calculate the statistic for every bootstrap sample.
    \item A histogram of the resulting values approximates the shape of the sampling distribution of the statistic.
  \end{itemize}
\end{frame}

\begin{frame}
\frametitle{Notation}
%\framesubtitle{} 
  \begin{itemize}
    \item Let $\mathbf{X} = (X_1, \ldots, X_n)$ be a random sample from some distribution $F$.
    \item $T=T(\mathbf{X})$ is a statistic (could be a vector of statistics).
    \item Form a ``bootstrap sample" $\mathbf{X}^*$ by sampling $n$ values from $\mathbf{X}$ \emph{with replacement}.
    \item Repeat this process $B$ times, obtaining $\mathbf{X}^*_1, \ldots, \mathbf{X}^*_B$
    \item Calculate the statistic for each bootstrap sample, obtaining $T^*_1, \ldots, T^*_B$
  \end{itemize}
\end{frame}

\begin{frame}
\frametitle{What can we do with $T^*_1, \ldots, T^*_B$?}
We can do lots of interesting things, including constructing confidence intervals (sometimes).

\vspace{5mm}

In this course, we will use it mostly to calculate standard errors the easy way.

\vspace{5mm}


  \begin{itemize}
    \item Approximate $Var(T)$ with the sample variance of the $T^*$ values.
    \item Or, if $T$ is a vector, approximate the asymptotic covariance matrix with the sample covariance matrix of the $T^*$ vectors.
    \item It can be a lot less work than the delta method.
  \end{itemize}
\end{frame}

\begin{frame}
\frametitle{Why does it work?}
% 
  \begin{itemize}
    \item Resampling from $\mathbf{X}$ is the same as simulation of a random variable whose distribution is the empirical distribution function $\widehat{F}(x)$. 
    \item $\widehat{F}(x)$ is the proportion of sample observations less than or equal to $x$.
    \item The empirical distribution function converges almost surely to the true distribution function by the Law of Large Numbers.
    \item Moments and lots of other useful quantities are smooth functions of the distribution function.
    \item So as $n\rightarrow\infty$ and $B\rightarrow\infty$, bootstrap sample moments of $T^*_1, \ldots, T^*_B$ converge to the corresponding moments of the sampling distribution of $T$.
  \end{itemize}
\end{frame}

\section{Distribution-free regression example}

\begin{frame}
\frametitle{Example: Distribution-free regression}
\framesubtitle{} 
Independently for $i=1, \ldots, n$, let
\begin{displaymath}
    Y_i = \beta_0 + \beta_1 X_i + \epsilon_i,
\end{displaymath}

where 

\begin{itemize}
     \item $X_i$ and $\epsilon_i$ come from unknown distributions
     \item $E(\epsilon_i=0)$, $Var(\epsilon_i=\sigma^2)$
     \item Moments of $X_i$ will be denoted $E(X)$, $E(X^2)$, etc.
     \item $X_i$ and $\epsilon_i$ are independent
\end{itemize}

\vspace{10mm}
Observable data consist of the pairs $(X_1,Y_1), \ldots, (X_n,Y_n)$.
\end{frame}

\begin{frame}
\frametitle{Estimation}
%\framesubtitle{} 
Estimate $\beta_0$ and $\beta_1$ as usual by

\begin{eqnarray*}
    \widehat{\beta}_0 
       & = & \frac{\sum_{i=1}^n(X_i-\overline{X})(Y_i-\overline{Y})}
                  {\sum_{i=1}^n(X_i-\overline{X})^2} \\ \\
       & = & \frac{\sum_{i=1}^n X_i Y_i - n \overline{X} \, \overline{Y}}
                  {\sum_{i=1}^n X_i^2 - n \overline{X}^2} \mbox{ and} \\ \\ \\
    \widehat{\beta}_1 & = & \overline{Y} -  \widehat{\beta}_0 \overline{X}
\end{eqnarray*} 

\begin{itemize}
     \item Consistency follows from the Law of Large Numbers and continuous mapping.
     \item Looks like $\widehat{\beta}_0$ and $\widehat{\beta}_1$ are asymptotically normal.
     \item Use this to get tests and confidence intervals.
\end{itemize}
\end{frame}

\begin{frame}
\frametitle{Set up the problem systematically}
%\framesubtitle{} 
Denoting the ``data vector" for case $i$ by $\mathbf{D}_i$,
{\small
\begin{displaymath}
    \mathbf{D}_i = 
    \left( 
    \begin{array}{c}
    X_i \\ X^2_i \\ Y_i \\ X_i Y_i 
    \end{array}
    \right), \mbox{ and~~ }
     \overline{\mathbf{D}}_n = 
    \left( 
    \begin{array}{l}
    \frac{1}{n}\sum_{i=1}^n X_i \\\\ \frac{1}{n}\sum_{i=1}^n X^2_i \\\\ 
    \frac{1}{n}\sum_{i=1}^n Y_i \\\\ \frac{1}{n}\sum_{i=1}^n  X_i Y_i 
    \end{array}
    \right) \stackrel{a.s.}{\rightarrow}
    \left(
    \begin{array}{l}
    E(X) \\\\ E(X^2) \\\\ 
    E(Y) \\\\ E(XY) 
    \end{array}
    \right) = \boldsymbol{\mu}
\end{displaymath} 
} % End size

Then $\left( 
    \begin{array}{c}
    \widehat{\beta}_0 \\ \widehat{\beta}_1
    \end{array}
    \right) = g(\overline{\mathbf{D}}_n) \stackrel{a.s.}{\rightarrow} g(\boldsymbol{\mu})
    = \left( 
    \begin{array}{c}
    \beta_0 \\ \beta_1
    \end{array}
    \right)$
\end{frame}

\begin{frame}
\frametitle{What would we do next}
\framesubtitle{To use the Central Limit Theorem and Delta method?} 
CLT says $\sqrt{n}(\overline{\mathbf{D}}_n-\boldsymbol{\mu}) \stackrel{d}{\rightarrow} 
    \mathbf{T} \sim N\left(\mathbf{0}, \boldsymbol{\Sigma} \right)$. Delta method says

 $\sqrt{n}(g(\overline{\mathbf{D}}_n)-g(\boldsymbol{\mu})) \stackrel{d}{\rightarrow} 
\mathbf{Y} \sim N(\mathbf{0},
\mbox{\.{g}}(\boldsymbol{\mu})\mathbf{\Sigma}\mbox{\.{g}}(\boldsymbol{\mu}) ^\prime)$, where

{\small
\begin{displaymath}
    \mathbf{D}_i = 
    \left( 
    \begin{array}{c}
    X_i \\ X^2_i \\ Y_i \\ X_i Y_i 
    \end{array}
    \right)
\end{displaymath} }

  \begin{itemize}
    \item Calculate $\boldsymbol{\Sigma}(\boldsymbol{\mu}) = V(\mathbf{D}_i)$ by hand.
    \item Calculate \.{g}$(\boldsymbol{\mu})$ by hand.
    \item Estimate $\boldsymbol{\Sigma}$ using sample moments, by computer.
    \item Calculate the asymptotic covariance matrix 
    $\frac{1}{n}\mbox{\.{g}}(\boldsymbol{\mu})
\widehat{\boldsymbol{\Sigma}}_n \mbox{\.{g}}(\boldsymbol{\mu}) ^\prime$ by computer
    \item Use that in confidence intervals and Wald-like tests.
  \end{itemize}
\end{frame}

\begin{frame}
\frametitle{It's a lot of work.}
%\framesubtitle{} 
  \begin{itemize}
    \item Most problems have more than one explanatory variable
    \item You could easily make a ``little" mistake.
  \end{itemize}
\end{frame}


\begin{frame}
\frametitle{Bootstrap approach: All by computer}
%\framesubtitle{} 
  \begin{itemize}
    \item Earlier discussion implies $\widehat{\boldsymbol{\beta}}$ is asymptotically normal.
    \item All we need is a good estimated covariance matrix of the sampling distribution.
    \item Put data vectors $\mathbf{D}_i = (\mathbf{X}_i,Y_i)^\prime$ in a jar.
    \item Sample $n$ vectors with replacement, yielding $\mathbf{D}_1^*$. Fit the regression model, obtaining $\widehat{\boldsymbol{\beta}}^*_1$.
    \item Repeat $B$ times. This yields $\widehat{\boldsymbol{\beta}}^*_1 \ldots \widehat{\boldsymbol{\beta}}^*_B$.
    \item The sample covariance matrix of $\widehat{\boldsymbol{\beta}}^*_1 \ldots \widehat{\boldsymbol{\beta}}^*_B$ is the estimated asymptotic covariance matrix we want.
  \end{itemize}
\end{frame}

\begin{frame}
\frametitle{Remark}
%\framesubtitle{} 
This is not a typical bootstrap regression.
  \begin{itemize}
    \item Usually people fit a model and then bootstrap the residuals, not the whole data vector.
    \item Bootstrapping the residuals applies to conditional regression (conditional on $\mathbf{X}=\mathbf{x}$).
    \item Our regression model is unconditional.
    \item The large-sample arguments are simpler in the unconditional case. 
  \end{itemize}
\end{frame}

%%%%%%%%%%%%%%%%%%%%%%%%%%%%%%%%%%%%%%%%%%%%%%%%%%%%%%%%%%%%%%%%%%%%%%%%%%%%%%%%%%%%%%%%%


\begin{frame}
\frametitle{Copyright Information}

This slide show was prepared by  \href{http://www.utstat.toronto.edu/~brunner}{Jerry Brunner},
Department of Statistics, University of Toronto. It is licensed under a 
\href{http://creativecommons.org/licenses/by-sa/3.0/deed.en_US}
     {Creative Commons Attribution - ShareAlike 3.0 Unported License}. Use any part of it as you like and share the result freely. The \LaTeX~source code is available from the course website:
\href{http://www.utstat.toronto.edu/~brunner/oldclass/appliedf12} {\small\texttt{http://www.utstat.toronto.edu/$^\sim$brunner/oldclass/appliedf12}}





\end{frame}


\end{document}


%%%%%%%%%%%%%%%%%%%%%%%%%%%%%%%%%%%%%%%%%%%%%%%%%%%%%%%%%%%%%%%%%%%%%%%%%%%%%%%%%%%%%%%%%

\begin{frame}
\frametitle{}
%\framesubtitle{} 
  \begin{itemize}
    \item 
    \item 
    \item 
  \end{itemize}
\end{frame}

{\LARGE
\begin{displaymath}
    
\end{displaymath} }




%%%%%%%%%%%%%%%%%%%%%%%%%%%%%%%%%%%%%%%%%%%%%%%%%%%%%%%%%%%%%%%%%%%%%%%%%%%%%%%%%%%%%%%%%



%%%%%%%%%%%%%%%%%%%%%%%%%%%%%%%%%%%%%%%%%%%%%%%%%%%%%%%%%%%%%%%%%%%%%%%%%%%%%%%%%%%%%%%%%
