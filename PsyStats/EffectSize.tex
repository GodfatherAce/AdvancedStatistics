%===============================================================================================%
\begin{frame}
\frametitle{SPSS : Partial Eta Squared}
\begin{itemize}
\item As we have mentioned before, we need to look at the effect size (strength of association) as well as statistical significance, so we can compare the strength of the effect of different variables (and across different studies). Our effect size index in ANOVA is called Eta squared ().  
\item Eta squared is the proporation of the total variance that is atrributed to an effect.
 (2)

\item Eta Squared varies between 0 and 1, and is interpreted in the usual way, i.e. 0 - .1 is a weak effect, .1 - .3 is a modest effect, .3 - .5 is a moderate effect and >.5 is a strong effect (remember, though that these cut-off points are just guidelines).
\end{itemize}
\end{framed}
%===============================================================================================%
\begin{frame}

Note that SPSS calculates the partial Eta squared. Partial effect size indicated the proportion of variance in the dependent variable explained by the independent variable.

%- http://stats.stackexchange.com/questions/15958/how-to-interpret-and-report-eta-squared-partial-eta-squared-in-statistically
