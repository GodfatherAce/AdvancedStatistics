% Title Page
\title{}
\author{}


\begin{document}
\maketitle

\section{Important Notation}
\begin{itemize}
\item $\mu$ The population mean.
\item $\bar{x}$ The sample mean of sample x.
\item $\sigma^2$ The population variance
\item $s^2$ The sample variance
\item $\sigma$ The population standard deviation
\item $s$ The sample deviation
\end{itemize}
\section{Introduction to Confidence Intervals}

\begin{itemize}
\item Confidence Intervals are constructed according to specified confidence levels. The most commonly used confidence level is 95\%.
\item Confidence Intervals are the most commonly used estimates from the family of \textbf{Interval Estimates}.
\item All confidence intervals have the same basic structure.
\[\mbox{Point Estimate} \pm \left( \mbox{Quantile} \times \mbox{Standard Error} \right)\]
\item Point Estimates are single value estimates. Most commonly used point estimates are:
\begin{itemize}
\item The Sample Mean $\bar{x}$ (usually pronouced as `x-bar')
\item The Sample Proportion $\hat{p}$ (usually pronouced as `p-hat')
\item The Difference of Two Sample Means $\bar{x}_1 - \bar{x}_2 $
\end{itemize}
\item Quantiles are values that are used to scale the confidence interval to the appropriate confidence level. Quantiles are usually found in statistical tables, such as the \textbf{Murdoch Barnes} tables. Quantiles are usually either one of
\begin{itemize}
\item \textbf{Z-value} : Murdoch Barnes Tables 3 
\item \textbf{t-values} : Murdoch Barnes Table 7
\end{itemize}
 We will look at quantiles in detail later.
\item There is a unique standard error for each type of point estimate. A list of standard error formulae is usually provided at the back of exam papers. For example,using the notation introduced in the previous section, he standard error for a sample mean is 
\[ S.E (\bar{x}) =  \frac{\sigma}{\sqrt{n}}\]


\end{itemize}

\end{document}          
