In optics, radiometry is a set of techniques for measuring electromagnetic radiation, including visible light. Radiometric techniques characterize the distribution of the radiation's power in space, as opposed to photometric techniques, which characterize the light's interaction with the human eye. Radiometry is distinct from quantum techniques such as photon counting.
Radiometry is important in astronomy, especially radio astronomy, and plays a significant role in Earth remote sensing. The measurement techniques categorized as radiometry in optics are called photometry in some astronomical applications, contrary to the optics usage of the term.
Radiosity is a global illumination algorithm used in 3D computer graphics rendering. Radiosity is an application of the finite element method to solving the rendering equation for scenes with purely diffuse surfaces. Unlike Monte Carlo algorithms (such as path tracing) which handle all types of light paths, typical radiosity methods only account for paths which leave a light source and are reflected diffusely some number of times (possibly zero) before hitting the eye. Such paths are represented as "LD*E". Radiosity calculations are viewpoint independent which increases the computations involved, but makes them useful for all viewpoints.
Radiosity methods were first developed in about 1950 in the engineering field of heat transfer. They were later refined specifically for application to the problem of rendering computer graphics in 1984 by researchers at Cornell University.
Trichromacy or trichromaticism is the condition of possessing three independent channels for conveying color information, derived from the three different cone types.[1] Organisms with trichromacy are called trichromats.
The normal explanation of trichromacy is that the organism's retina contains three types of color receptors (called cone cells in vertebrates) with different absorption spectra. In actuality the number of such receptor types may be greater than three, since different types may be active at different light intensities. In vertebrates with three types of cone cells, at low light intensities the rod cells may contribute to color vision, giving a small region of tetrachromacy in the color space.

