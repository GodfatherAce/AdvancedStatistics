\section*{Beverton–Holt model}
%======================================================================================================%
The Beverton–Holt model is a classic discrete-time population model which gives the expected number n t+1 (or density) of individuals in generation t + 1 as a function of the number of individuals in the previous generation,

\[n_{t+1} = \frac{R_0 n_t}{1+ n_t/M}. \]
Here R0 is interpreted as the proliferation rate per generation and K = (R0 − 1) M is the carrying capacity of the environment. The Beverton–Holt model was introduced in the context of fisheries by Beverton & Holt (1957). Subsequent work has derived the model under other assumptions such as contest competition (Brännström & Sumpter 2005) or within-year resource limited competition (Geritz & Kisdi 2004). 

The Beverton–Holt model can be generalized to include scramble competition (see the Ricker model, the Hassell model and the Maynard Smith–Slatkin model). It is also possible to include a parameter reflecting the spatial clustering of individuals (see Brännström & Sumpter 2005).

Despite being nonlinear, the model can be solved explicitly, since it is in fact an inhomogeneous linear equation in 1/n. 

The solution is

\[n_t = \frac{K n_0}{n_0 + (K - n_0) R_0^{-t}}.\]
Because of this structure, the model can be considered as the discrete-time analogue of the continuous-time logistic equation for population growth introduced by Verhulst; for comparison, the logistic equation is

\[\frac{dN}{dt} = rN \left( 1 - \frac{N}{K} \right),\]
and its solution is


\[N(t) = \frac{K N(0)}{N(0) + (K - N(0)) e^{-rt}}.\]

%======================================================================================================%
\end{document}
