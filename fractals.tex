\subsection*{Fractal}
%===================%
An object in which the parts are in some way related to the whole. That is, the individual components are "self-similar." An example is the branching network in a tree. While each branch, and each successive smaller branching is different, they are qualitatively similar to the structure of the whole tree.

\subsubsection*{Fractal Distribution}
A probability density function that is statistically self-similar. That is, in different increments of time, the statistical characteristics remain the same.

\subsubsection*{Self-Similarity}
When small parts of an object are qualitatively the same, or similar to the whole object. In certain deterministic fractals, like the Sierpinski Triangle, small pieces look the same as the entire object. In random fractals, small increments of time will be statistically similar to larger increments of time. See: Fractal.

\subsubsection*{ Websites}
http://www.cs.sunysb.edu/~skiena/691/lectures/lecture12.pdf
