% - http://rss.org.uk/uploadedfiles/userfiles/files/Syllabus%202015.pdf
\subsection*{Probability}
\begin{itemize}	
\item Sampling with and without replacement. 
\item Elementary problems involving urn models.
\item Joint probability, marginal and conditional probability, independence.
\item Law of total probability. 
\item Bayes' Theorem.
\end{itemize}

%------------------------------------------------------------------------------------------%
\subsection*{Distribution theory}

\begin{itemize}	
\item Random variables.
\item Discrete and continuous random variables. 
\item The probability mass function and probability density function. 
\item Cumulative  distribution function. 
\item Expectation as a linear operator. 
\item Expectation of functions of a random variable. 
\item Mean and variance.
\item Approximate mean and variance of a function of a random variable. 
\item Variance-stabilising transformations
\end{itemize}
\subsection*{Distribution theory : Distributionss}
Standard distributions and their use in modelling, including Bernoulli, binomial, Poisson, geometric, negative binomial, hypergeometric, discrete uniform, Normal, exponential, gamma, contin-uous uniform, beta, Weibull, Cauchy, lognormal.

%-------------------------------------------------------------------------------------------%

\begin{itemize}	
\item	Distributions of functions of several random variables. 
\item	Transformations, including the probability integral transform. 
\item	Joint distribution of mean and variance from a Normal random sample.
\end{itemize}

% - http://www.ece.tufts.edu/~maivu/ES150/5-mrv_func.pdf
% - en.wikipedia.org/wiki/Probability_integral_transform
% - www.maths.lancs.ac.uk/~nealp/pitslides.pdf

%-------------------------------------------------------------------------------------------%
\subsection*{Simulation}
\item Generation of uniform pseudo-random numbers; 
testing for uniformity.
\item Methods of generating random numbers from common distributions, including inversion, 
rejection and table look-up techniques. 
\item Monte Carlo methods. 
\item Use of variance reduction techniques. 
\item Applications of simulation.
\end{itemize}
%-------------------------------------------------------------------------------------------%

\subsection*{Variance-Stabilizing Transformation}
In applied statistics, a variance-stabilizing transformation is a data transformation that is specifically chosen either to simplify considerations in graphical exploratory data analysis or to allow the application of simple regression-based or analysis of variance techniques.[1]
 
The aim behind the choice of a variance-stabilizing transformation is to find a simple function $f$ to apply to values x in a data set to create new values y = ƒ(x) such that the variability of the values y is not related to their mean value. 
%-------------------------------------------------------------------------------------------%

\subsection*{Anscombe Transformation}
In statistics, the Anscombe transform, named after Francis Anscombe, is a variance-stabilizing transformation that transforms a random variable with a Poisson distribution into one with an approximately standard Gaussian distribution. The Anscombe transform is widely used in photon-limited imaging (astronomy, X-ray) where images naturally follow the Poisson law. The Anscombe transform is usually used to pre-process the data in order to make the standard deviation approximately constant. Then denoising algorithms designed for the framework of additive white Gaussian noise are used; the final 
estimate is then obtained by applying an inverse Anscombe transformation to the denoised data.
