MODULE 5: Topics in applied statistics

This module covers four application areas, denoted by bold sub-headings below. It introduces some general techniques for data analysis and some that are more specialised to particular areas. The paper for this module will not be formally divided into sections but it will always consist of two questions on each of the four application areas covered by the syllabus. Candidates are particularly advised that some questions on the paper are likely to require knowledge of material covered in the module on Modelling Experimental Data (module 4).
Critical evaluation of computer output is required throughout this module.
Note that this syllabus is continued on the next page.
 

MODULE 5: Topics in applied statistics
Multivariate methods
Censored data: survival and reliability
Demography and epidemiology
Sampling
Multivariate methods
Vectors of expected values. Covariance and correlation matrices.
Discriminant analysis, choice between two populations, calculation of discriminant function, and probability of misclassification, test and training samples, leave-one-out and k-fold cross-validation, idea of extension to several populations.
Principal components; definition, interpretation of calculated components, use in regression.
Cluster analysis, similarity measures, single-link and other hierarchical methods, k-means.
Informal approaches to checking for multivariate Normality. Tests and confidence regions for multivariate means.

Censored data: survival and reliability
Problems involving censored data, for example in clinical and engineering contexts.
Reliability and life testing.
Hazard and survivor functions.
Kaplan-Meier estimate of survivor function.
Weibull and hazard plots.
Confidence intervals for survivor function using Greenwood's formula.
Logrank test.
Parametric survival distributions - exponential, Weibull.

An understanding of the assumptions and interpretation of the fitted model. Details of partial likelihood and numerical methods for fitting the model not required. Calculation and interpretation of hazard ratios and confidence intervals.
Checking for non-proportionality of hazards.
Checking for non-proportional hazards using a log cumulative hazard plot and plots of hazard functions.
Demography and epidemiology
Population pyramids.
Life tables.
Standardised rates (e.g. mortality).
Construction and use of life tables; derived quantities, including calculation of life expectancy. Direct and indirect standardisation.
Incidence and prevalence.
Design and analysis of cohort (prospective) studies.
Design and analysis of case-control (retro-spective) studies.
Confounding and interaction.
Matched case control design and analyses, using McNemar's test.
Causation.
Relative risk. Odds ratio. Estimation and confidence intervals for 2×2 tables.
Mantel-Haenszel procedure.
Sensitivity, specificity, ROC curves, positive predictive value, negative predictive value.
Distinction between these concepts.
Reasons for matching; advantages and dis-advantages relative to unmatched studies.
Inferring causality from observational studies.
Use in adjusting for confounding variables.
Uses in screening and diagnosis.
Sampling
Census and sample survey design. Target and study populations, uses and limitations of non-probability sampling methods, sampling frames, sampling fraction.
Revision and extension of basic concepts from Higher Certificate.
Simple random sampling. Estimators of totals, means and proportions; bias. Estimated standard errors, confidence intervals and precision. Sampling fraction and finite population correction. Ratio and regression estimators.
Examples of practical use in various contexts.
Use of supplementary information.
Stratified random sampling. Estimators of totals, means and proportions; bias. Estimated standard errors, confidence intervals and precision. Cost functions. Proportional and optimal allocations. Limitations of stratified sampling.
Examples of practical use in various contexts.
Minimisation of cost × variance.
One-stage cluster sampling. Estimators for totals, means and proportions with equal cluster sizes and with different cluster sizes. Estimated standard errors, confidence intervals and precision. Link with systematic sampling. Description of two-stage sampling and of multi-stage sampling. Limitations.
Examples of practical use in various contexts.
