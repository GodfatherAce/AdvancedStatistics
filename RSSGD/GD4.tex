\section*{MODULE 4: Modelling experimental data}
This module covers the general and generalised linear model. 
It also covers the design of experiments and analysis of experimental results using analysis of variance and multiple regression, extending the coverage of these topics in the Higher Certificate. Candidates are expected to have knowledge of the use of these methods and the interpretation of results from them, including computer output: critical evaluation of computer output is required throughout this module.
%------------------------------------------------------------------%
\subsection*{General linear model}
\begin{itemize}
\item	Least squares. 
\item	Properties of LS estimators in the linear model, Gauss-Markov theorem. 
\item	Models for simple and multiple regression and for analysis of variance and covariance. 
\item	Hat matrix. 
\item	Estimation of variance. 
\item	Interval estimates of parameters. 
\item	Weighted least squares. 
\item	Importance of assumptions. 
\item	Analysis of residuals. 
\item	Transformation of variables. 
\item	Linearising other models, e.g. multiplicative models and growth curves.
\item	Proof of standard results for multiple regression (matrix notation; matrix differentiation will not be required). 
\item	Application of these to other situations as listed.
\end{itemize}
%------------------------------------------------------------------%
\subsection*{Design of experiments}
\begin{itemize}
\item	Randomisation, replication, blocking.
\item	Completely randomised designs, randomised blocks, Latin squares, balanced incomplete blocks.
\item	Factorial treatment structure.
\item	$2^n$ designs, including confounding and fractional replication.
\item	Reasons for using these designs; bias and precision. 
\item	How to construct a valid randomised layout for each design. 
\end{itemize}

Knowledge of what balanced incomplete block designs exist for reasonably small block sizes and numbers of treatments.

%------------------------------------------------------------------%
\subsection*{Analysis of variance}
\begin{itemize}
\item	Analysis of variance for the designs listed above, including for cross-classifications with replication, and for nested or hierarchical designs. Fixed and random effects; variance components. Application to data collected in experiments or by sampling.
\item	General linear contrasts among treatments. 
\item	Inference for individual treatment means and for contrasts.
\item	Analysis of residuals. 
\item	Use of plotting techniques to detect non-Normality of errors.
\item	Estimation of variance components, and use in planning sampling schemes.
\item	Comparisons between and within groups of treatments. 
\item	Linear and quadratic components for factors at equally spaced levels.
\end{itemize}
%------------------------------------------------------------------%
\subsection*{Multiple regression}
\begin{itemize}
\item	Regression with more than one explanatory variable. 
\item	Use of indicator variables to represent factors. 
\item	Analysis of variance of regression, including test for lack of fit. 
\item	Analysis of residuals, regression diagnostics, detection of influential observations, multicollinearity, serial correlation. 
\item	Model selection, using all-subset and stepwise methods.
\item	Extension to non-linear modelling. 
\item	Fitting standard growth curves (including logistic and Gompertz). 
\item	Estimation of parameters, approximate confidence intervals and tests.
\item	The Extra Sum of Squares principle.
\item	Pure error.
\item	Concept of leverage. 
\item	$R^2$ and adjusted $R^2$. Use of the Durbin-Watson statistic.
\item	Including use of $C_p$ plots.
\end{itemize}

Candidates will be expected to have some familiarity with the Newton-Raphson procedure for fitting non-linear models.

%------------------------------------------------------------------------------------------------%
\newpage
\subsection*{The Latin Square Design}
The Latin square design is used where the researcher desires to control the variation in an experiment that is related to rows and columns in the field.

Field marks: 
\begin{itemize}
\item Treatments are assigned at random within rows and columns, with each treatment once per row and once per column. 
\item There are equal numbers of rows, columns, and treatments. 
\item Useful where the experimenter desires to control variation in two different directions 
\end{itemize}
%------------------------------------------------------------------------------------------------%
\newpage
\subsection*{Durbin Watson Statistic}

A number that tests for autocorrelation in the residuals from a statistical regression analysis. 
The Durbin-Watson statistic is always between 0 and 4. 
A value of 2 means that there is no autocorrelation in the sample. 
Values approaching 0 indicate positive autocorrelation and values toward 4 indicate negative autocorrelation. 

\end{document}
