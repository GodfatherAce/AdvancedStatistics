The probability density function of a Weibull random variable is:

\[ \mbox{Equation Here} \]
 
where k > 0 is the shape parameter and $\lambda$ > 0 is the scale parameter of the distribution. Its complementary cumulative distribution function is a stretched exponential function. The Weibull distribution is related to a number of other probability distributions; in particular, it interpolates between the exponential distribution (k = 1) and the Rayleigh distribution (k = 2).
 
If the quantity X is a "time-to-failure", the Weibull distribution gives a distribution for which the failure rate is proportional to a power of time. 

The shape parameter, k, is that power plus one, and so this parameter can be interpreted directly as follows:
\begin{itemize}
\item A value of k < 1 indicates that the failure rate decreases over time. This happens if there is significant "infant mortality", or defective items failing early and the failure rate decreasing over time as the defective items are weeded out of the population.
\item A value of k = 1 indicates that the failure rate is constant over time. This might suggest random external events are causing mortality, or failure.
\item A value of k > 1 indicates that the failure rate increases with time. This happens if there is an "aging" process, or parts that are more likely to fail as time goes on.
\end{itemize}
%------------------------------------------------------------------------------------------------%
\subsubsection*{Application}
The Weibull distribution is used
\begin{itemize}
\item	 In survival analysis
\item	 In reliability engineering and failure analysis
\item	 In industrial engineering to represent manufacturing and delivery times
\item	 In extreme value theory
\item	 In weather forecasting To describe wind speed distributions, as the natural distribution often matches the Weibull shape
\item	 In communications systems engineering In radar systems to model the dispersion of the received signals level produced by some types of clutters
\item	 To model fading channels in wireless communications, as the Weibull fading model seems to exhibit good fit to experimental fading channel measurements
\item	 Fitted cumulative Weibull distribution to maximum one-day rainfalls using CumFreq, see also distribution fitting In general insurance to model the size of reinsurance claims, and the cumulative development of asbestosis losses
\item	 In forecasting technological change (also known as the Sharif-Islam model)
\item	 In hydrology the Weibull distribution is applied to extreme events such as annual maximum one-day rainfalls and river discharges. The blue picture illustrates an example of fitting the Weibull distribution to ranked annually maximum one-day rainfalls showing also the 90% confidence belt based on the binomial distribution. The rainfall data are represented by plotting positions as part of the cumulative frequency analysis.
\item	 In describing the size of particles generated by grinding, milling and crushing operations, the 2-Parameter Weibull distribution is used, and in these applications it is sometimes known as the Rosin-Rammler distribution.[citation needed] In this context it predicts fewer fine particles than the Log-normal distribution and it is generally most accurate for narrow particle size distributions.[citation needed] The interpretation of the cumulative distribution function is that F(x; k; λ) is the mass fraction of particles with diameter smaller than x, where λ is the mean particle size and k is a measure of the spread of particle sizes.
\item	 Application of these to other situations as listed.
\end{itemize}
%------------------------------------------------------------------------------------------------%
