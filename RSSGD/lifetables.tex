 
Life Tables
 
In actuarial science, a life table (also called a mortality table or actuarial table) is a table which shows, for each age, what the probability is that a person of that age will die before his or her next birthday. From this starting point, a number of inferences can be derived.
 
the probability of surviving any particular year of age
remaining life expectancy for people at different ages
Mathematics of Life Tables
 
qx: the probability that someone aged exactly x will die before reaching age (x+1).
px: the probability that someone aged exactly x will survive to age (x+1)
 
px=1-qx
 
lx the number of people who survive to age x
 
Note that this is based on a starting point of l0 lives, typically taken as 100,000
 
 
        lx + 1=lx(1-qx) =lxpx            also  lx + 1lx=px
 
dx the number of people who die aged x last birthday
        dx=lx-lx+1=lx(1-px) =lxqx 
 
tpx the probability that someone aged exactly x will survive for t more years, i.e. live up to at least age x+t years
        tpx=lx+tlx 
 
 t|kqx: the probability that someone aged exactly x will survive for t more years, then die within the following k years
        t|kqx=tpxkqx+t=lx+t-lx+t+klx 
 
 
x: the force of mortality, ie the instantaneous mortality rate at age x, ie the number of people dying in a short interval starting at age x, divided by lx and also divided by the length of the interval.
 
Unlike qx, the instantaneous mortality rate, x, may exceed 1.
 
mx  This symbol refers to Central rate of mortality. It is approximately equal to the average force of mortality, averaged over the year of age.
 
 
Actuaries September 2011 CT6 Question 9
(i) State the principle of correspondence as it applies to the estimation of mortality rates. 
(ii) Explain why it might be difficult to ensure the principle of correspondence is adhered to, and give a specific example of an investigation where this may be
the case. 
An actuary was asked to investigate the mortality of lives in a particular geographical area. Data are available of the population of this area, classified by age last birthday,
on 1 January in each year. Data on the number of deaths in this area in each calendar year, classified by age nearest birthday at death, are also available.
(iii) Derive a formula which would allow the actuary to estimate the force of mortality at age x + f, x+ f , in a particular calendar year, in terms of the
available data, and derive a value for f. [6]
(iv) List four factors other than geographical location which a government statistical office might use to subdivide data for national mortality analysis. [2]

Actuaries April 2011 CT6 Question 6
A study of the mortality of a certain species of insect reveals that for the first 30 days of life, the insects are subject to a constant force of mortality of 0.05. After 30 days,
the force of mortality increases according to the formula:

30+x=0.05exp(0.01x), 

where x is the number of days after day 30.

(i) Calculate the probability that a newly born insect will survive for at least 10 days. [1]
(ii) Calculate the probability that an insect aged 10 days will survive for at least a further 30 days. [3]
(iii) Calculate the age in days by which 90 per cent of insects are expected to havedied. [4]



